\documentclass[letterpaper]{article}
%\documentclass[letterpaper,10pt]{scrartcl}

\usepackage[utf8]{inputenc}
\usepackage[pass,letterpaper]{geometry}
\usepackage{graphicx}
\usepackage{listings}

\setlength{\parindent}{0cm}

\title{Non-Blocking In-Memory Database\thanks{Sponsor: Dr. Damian Dechev}}
\author{Michael McGee, Robert Medina, Neil Moore, Jason Stavrinaky\\[2ex]
	\includegraphics{graphics/OpenMemDB_logo_transparency.png}\\[1ex]
}
\date{11/4/2015}

\pdfinfo{%
  /Title    (Non-blocking, In-memory Database)
  /Author   (Michael McGee, Robert Medina, Neil Moore, Jason Stavrinaky)
  /Creator  (Neil Moore)
  /Producer ()
  /Subject  (Final Document for COP4934)
  /Keywords ()
}

\begin{document}
\pagenumbering{gobble}
\maketitle
\newpage

\pagenumbering{roman}
\tableofcontents
\newpage

\pagenumbering{arabic}

\section{Executive Summary}
The database as a concept has always emphasized speed, as many other pieces of both software
and business rely on the information contained within them. As such, these pieces of software
are part of a class of software that strive for every piece of performance possible, either through
hardware improvements or algorithmic optimizations. While in-memory databases are not a new concept,
Random Access Memory (RAM) has, until recently, not been cheap enough to completely house a useful
set of data. The ability to do this in such a way beyond simply using main memory as a cache for the 
actual data on the hard disk opens up the possibility of fully utilizing the processing power
available to the system fully.
\par\vspace{\baselineskip}
Utilizing the full hardware resources available to us requires the use of multiple threads or 
multiple processes spread out over all the cores available. Accessing the same data location 
from multiple threads or processes causes issues where threads are unable to progress further in their
work. This state is called deadlock, and is the bane of any programmer that creates multi-threaded
programs. Our project attempts to entirely avoid both deadlock and their common mitigator, locks,
by implementing and using wait-free data structures and algorithms.
\par\vspace{\baselineskip}
Wait-free is a guarantee that the system will always progress, as a whole, within a given period of time
regardless of work-load or contention over resources. The application of this guarantee to many
of the common algorithms and data structures familiar to computer scientists is still under heavy research
(as an example, there is still no widely-accepted implementation of a wait-free binary search tree).
\par\vspace{\baselineskip}
The objectives for this project is to successfully implement a SQL database that is both fully in-memory
and fully wait-free, even at the cost of performance. In order for this to be possible within the time
given to us, the scope of this project must be toned down and strictly enforced. While we will 
attempt to be SQL-compliant, certain aspects of that standard heavily imply a type of mutual 
exclusion be used which we obviously cannot use if we wish to remain wait-free. The pure size of
the standard is also a major obstacle to this objective.
\par\vspace{\baselineskip}
While our technical approach is prone to change as our understanding of the various concepts
and systems at play in a Database Management System (DBMS), we do have a general sense that we
wish to utilize a functional or imperative style of data flow. Objects will be used when appropriate
but we would not describe the overall design as object-oriented in any way. On the contrary, we would 
find a much more suitable term in data-oriented programming as we are almost entirely dealing
with large amounts of arbitrary manipulation rather than the interaction of objects.
\par\vspace{\baselineskip}
To facilitate a wait-free system, we will implement a system where a pre-determined pool of threads
that will be assigned a queue of tasks, whether they be SQL queries or database-specific commands.
These threads will be almost entirely distinct and independent of the others, with as little
inter-communication as possible in order to avoid the possibility of deadlock or the necessity of
locking. The assignment of these tasks will be done by a work manager that will perform some form 
of load-balancing when distributing the tasks among the pool of threads.
\par\vspace{\baselineskip}
Each thread will independently analyze, plan and execute its tasks in such a way that no shared memory
outside of the actual data is needed. The access to the data store, necessary in any database, 
will be handled via a common set of interfaces that will then utilize the algorithms and data structures
given to us by our sponsor. The planning and optimization of these tasks will be primary source 
of technical challenge in this project as the task language chosen (SQL) is declarative rather
than imperative or functional in nature. In other words, the tasks' language only tells us what
to retrieve rather than how to retrieve it.
\par\vspace{\baselineskip}
The advantages given to us by the approach detailed above is the inherent and explicit wait-freedomn
that occurs when the need for locking or shared state is removed entirely. As a result of that, we 
hope to be at least competitive or comparable to current in-memory or traditional databases in terms
of performance.

\newpage

\section{Project Motivation}
As hardware reaches the limits in Moore's Law and processors stop becoming faster and faster
and instead focus on becoming more and more parallel, algorithms and the software that implement
them must adapt in order to maximize both performance and hardware utilization.
\par\vspace{\baselineskip}
% Graphic showing Moore's Law tapering off (graph of GHz and number of cores?)
%\includegraphics{graphics/Moores_Law_core_freq_comparison.png}
Recent research done by multiple universities and companies have yielded the concept of wait-freedom,
the guarantee that the entire system will make progress towards a goal in a given period of time.
This is a stronger guarantee than lock-freedom as lock-freedom only guarantees that a single thread
will make progress in a period of time. Our sponsor, Dr. Dechev, and his lab here at the University
of Central Florida have created a framework of wait-free data structures named Tervel ~\cite{tervel}.

\subsection{Personal Motivations}
\subsubsection{Neil Moore}
This project is one that I knew would be incredibly challenging, yet also had the greatest
potential in terms of personal growth and end product. These technologies are only going to
become more and more important and widespread throughout the software industry as hardware 
continues to parallelize, memory becomes faster and cheaper, and wait-free/lock-free algorithms
become more mature. Experience working with high-performance, memory-intensive, and
standard-defined technologies opens the door for later opportunities in cutting-edge technologies.
\newpage

\section{Broader Impacts}
\newpage

\section{Specification and Requirements}
\newpage

\section{Research}
\subsection{Work Manager}
The Work Manager is a conglomeration of multiple minor, but not insignificant, roles
within the DBMS. Specifically, it is the:
\begin{itemize}
  \item Main thread
  \item Sole communicator with the client
  \item Load-balancer between the primary worker threads
  \item System resource monitor
\end{itemize}

Work distribution between threads is done via a task-queue that each thread in the
thread pool pulls work from. Tasks are C++11 lambda functions that are packaged up 
via the standard library's \lstinline|packaged\_task| class. This class provides an interface that
allows us to create a future

\subsection{Data Store}
\subsubsection{Indexing}
%author Robert Medina
Indexing allows for fast retrieval from a collection of data. There are many ways to accomplish this, and some ways are better suited depending on the constraints. Tree-based indexing and Hash-based indexing are two popular solutions for an implementation of a database. Given a set of data in memory, indexing takes a key data value and stores it into a data structure. Based off this key value, the indexing data structure will point to the file in memory or memory address, depending on how the data is stored. Based on this, a file scan of the database is now just reduced to scanning an index file. 
%\includegraphics{graphics/indexentry.png}
However searching through an index file can still be a costly operation. Index files are smaller than the data it is referencing but it can still use a considerable about memory space. Therefore it would be reasonable to use a data structure for fast retrieval of data based on a range of values or based on the actual data value stored.


\subsubsection{Tree-based Indexing}
%author Robert Medina
Tree-based indexing is an index file structured into a variant of a binary search tree. The tree is based off the key value and references to where that key is stored. Indexing requires fast retrieval of data, low cost insertion and low cost deletion of data. There are two types of trees that are useful for this type of operation, ISAM and B+ Tree.

\subsubsection{B+ Tree}
%author Robert Medina
B+ Tree is a balanced tree in which the non-leaf nodes direct where the data will be stored and the leaf nodes contain the data entries. The leaf nodes are contained in a doubly-linked list for fast retrieval of all leaf nodes.
%\includegraphics{graphics/B+_Tree_Abstract.png}
Specifically, B+ Tree is an n-ary tree that has a root, internal nodes and leaf children. Each index entry contains an array of key values and a pointer. Each node references to another node based on a branching property. Each parent node contains m+1 children, m is index entries. The left children must be less than the parent key value, the middle children must be between the left and right side children values, and the right children must be greater or equal to the parent key value. Every node except root must be at least half full. 
%\includegraphics{graphics/B+_Tree_Example.png}
The runtime for a B+ Tree is the following:
Searching is logm(n), m is index entries
Insertion is same as searching
Deletion is same as searching
Searching with range (logm(n) + k), m is index entries and k is number of data records

\subsubsection{Hash-based Indexing}
%author Robert Medina
	Hash-based Indexing references a key value to a pointer using a hash function. This can be used for indexing when there is an equal key value within the hash. Range operations using a key value is not possible with a hash function since it would simply be too costly of an operation. Hash-based indexing suffers from overflow chaining such as ISAM, which can hinder performance. There exists multiple hash-based indexing such as Static Hashing, Extendible Hashing, and Linear Hashing. 
\subsubsection{Static Hashing}
%author Robert Medina
Static Hashing is a hash table based on a key value with a pointer to a bucket of pages that contain said key. These data entries may be sorted but it depends on the application. This data structure is static for the most part but it does allow for overflow page allocation. In the event of inserting beyond the memory allocated, this data entry is placed into a new page and the page is added to an overflow chain in the hash table.
%\includegraphics{graphics/Static_Hashing_Abstract}
A good hash function is imperative to uniformly distribute values over the collection of buckets. An example of a good hash would be hash(key) = (a*key + b), a and b are constants. Some problems of Static Hashing are the fact that it is static. When the index file is created bucket sizes are known at time of creation, so pages can be stored successively in the buckets. However as the index file continues to grow if the same key value is stored repeatedly then a long overflow chain develops. Since the number of buckets are static if the index file shrinks in size then there is wasted memory space. If the file grows too large then it results in poor performance. Otherwise, the performance for operations is very fast. The following is the runtimes:
Search O() 1 I/O read
Insertion O() 2 I/O read/write page
Deletion O() 2 I/O read/write page
Rehashing:
	Intuitively a simple hash table with a pointer to a page would make sense. However in the case of additional pages, without an overflow chain rehashing the table would be necessary. In this case rehashing the table would be a costly operation. Including that the data structure is unusable while rehashing is in progress. Dynamic hashing techniques solve this problem.
Extendible Hashing:

\subsection{SQL Engine}
%author Mike McGee
One of the most important pieces of a Database Management System is the SQL Engine. 
This component is responsible for receiving commands written in the standard query
language and transforming those commands into an internal representation that can be
executed by the DBMS. The SQL Engine often consists of three pieces: the tokenizer, the
parser, and the code generator.

%\includegraphics{graphics/SQLEngine_activity_overview.png}

\subsubsection{Tokenizer}
%initial author Neil Moore
To start the query planning and execution process, we must tokenize the provided query
from a straight string into a format readable by the parser. This tokenizer is very
similar to a generic compiler tokenizer though it's job is made considerably easier 
by the importance of spaces in SQL queries/commands regarding token delineation.
\par\vspace{\baselineskip}
In an effort to reduce code duplication and facilitate rapid development, we chose
to appropriate the tokenizer from the SQLite project. SQLite is a public domain
SQL database library that integrates very cleanly into many applications big
or small.
\par\vspace{\baselineskip}
SQLite is not entirely SQL compliant, as that is simply unreasonable to achieve
given its use cases and problem domain. As such, changes to the tokenizer will be
needed in order to make sure it can handle proper SQL queries/commands while 
not supporting non-standard SQL tokens.

\subsubsection{Parser}
%author Mike McGee
When researching SQL parsers we found that most database management systems use a
parser generator tool to develop a parser for the query language that they support.
The purpose of any parser generator is to implement a parser in the programming
language desired that will accurately parse the context free grammar that it 
was passed in the grammar specification file.
\par\vspace{\baselineskip}
The two parser generators that we found were "Lemon" and "YACC". PostgreSQL uses 
"YACC", which stands for "Yet Another Compiler Compiler", while SQLite uses 
"Lemon", which stands for "Lemon". Both tools will generate a C code parser for 
your query language when provided a grammar specification file. According to the 
tutorial provided by "Lemon" there are some vast differences between these 
two parser generators. 
\par\vspace{\baselineskip}
“It uses a different grammar syntax which is designed to reduce the number of coding
errors. Lemon also uses a more sophisticated parsing engine that is faster than yacc and
bison and which is both reentrant and thread-safe. Furthermore, Lemon implements features
that can be used to eliminate resource leaks, making is suitable for use in long-running
programs such as graphical user interfaces or embedded controllers.”\cite{lemon_parser}
\par\vspace{\baselineskip}
It is because of these benefits, especially thread safety, that we chose to use Lemon
as our parser generator. The Lemon parser generator is contained in one C code file and
is used by running the program with the grammar specification file as an argument.
This terminal command would resemble \textit{lemon gram.y} 
Upon completion Lemon will produce between one and three files.\\ Those files are:
\begin{itemize}
	\item \textit{gram.c}: C code implementation of the parser
	\item \textit{gram.h}: A header file defining an integer ID for each terminal sybmol
	\item \textit{gram.out}: An information file that describes the states of the
	generated parser automaton
\end{itemize}
Lemon does not generate a complete program, it only creates a few subroutines that
implement a parser. It is up to the developer to call those subroutines in an appropriate
way in order to produce a complete system. In order to use a Lemon generated parser the
developer must first create the parser as follows: 
\begin{lstlisting} 
	void *pParser = ParseAlloc( malloc );
\end{lstlisting}
This call allocates and initializes a new parser and returns a pointer to it. The
parameter to the call is the subroutine used to allocate memory. For our purposes it will
most likely be something Tervel specific.

After the programmer is done using the parser they must free the memory that was allocated
to the parser using a subroutine of their choice. It is done as follows
\begin{lstlisting}
	ParseFree (pParser, free) 
\end{lstlisting}
where free is the subroutine used to reclaim the memory, again probably Tervel specific for our purposes. 

After the parser is allocated, the developer will provide the parser with a sequence of
tokens to be parsed. This is done by calling:
\begin{lstlisting}
	Parse(pParser, hTokenID, sTokenData, pArg);
\end{lstlisting}

% What's the deal with this? - Neil
\begin{quotation}
“The first argument to the Parse() routine is the pointer returned by ParseAlloc(). The
second argument is a small positive integer that tells the parse the type of the next
token in the data stream. There is one token type for each terminal symbol in the grammar.
The gram.h file generated by Lemon contains \#define statements that map symbolic terminal
symbol names into appropriate integer values. (A value of 0 for the second argument is a
special flag to the parser to indicate that the end of input has been reached.) The third
argument is the value of the given token. By default, the type of the third argument is
integer, but the grammar will usually redefine this type to be some kind of structure.
Typically the second argument will be a broad category of tokens such as ``identifier'' or
``number'' and the third argument will be the name of the identifier or the value of the
number. The Parse() function may have either three or four arguments, depending on the
grammar. If the grammar specification file request it, the Parse() function will have a
fourth parameter that can be of any type chosen by the programmer. The parser doesn't do
anything with this argument except to pass it through to action routines. This is a
convenient mechanism for passing state information down to the action routines without
having to use global variables.”\cite{lemon_parser}
\end{quotation}

<<<<<<< HEAD
\subsubsection{Query Planner}
The query planner is the most challenging and important piece of a typical DBMS as it
determines how the data in the SQL tables are retrieved which has a huge impact on 
the overall performance of the DBMS. The need for a query planner comes from the 
declarative nature of SQL queries, where the query doesn't tell you how to retrieve the
data but instead tells you what to retrieve. This is similar to languages such as Prolog where
a theorem solver is employed to determine how to solve the given problem. A large theorem 
solver is impractical in a system with high-performance, low-latency environment such as 
a DBMS, so we must generate our own significantly stripped-down version that can handle a 
specific set of query relations extremely fast.
=======
\subsubsection{Code Generator}

\subsection{Related Works}
	Over the last 30 years the there has been tremendous advancements in computing
	hardware. "Processors are thousands of times faster and memory is thousands of
	times times larger"\cite{stonebraker2007end}. Most technologies have advanced 
	along with hardware, however database management systems have struggled to improve
	at a similar rate. This is mostly due to concurrency issues. "Existing studies show
	that current database engines can spend more than 30\% of time in 
	synchronization-related operations (e.g.locking and latching), even when only a 
	single client thread is running."\cite{soares2015database}
   \par\vspace{\baselineskip}
  	There have been several attempts to solve this problem. Some of which will sacrifice
  	some data consistency in order to achieve a higher better performance. Still others
  	remain fully ACID compliant and attempt to parallelize individual steps in the 
  	DBMS or solely use multiple threads when executing query plans. Then there are those
  	that attempt to implement some level of lock freedom into their DBMS.
  	\par\vspace{\baselineskip}
	\subsubsection{MemSQL and VoltDB}
	MemSQL\footnote{MemSQL can be configured as a Columnstore that stores data on disk}
	and VoltDB are both fully in memory DBMS as is OpenMemDB. This is where the
	similarities end as far as OpenMemDB is concerned. MemSQL and VoltDB on the other 
	hand both use distributed systems to achieve performance gains. MemSQL differs from 
	VoltDB in a few ways, the most important being it's use of lock free data structures
	for storing data and its storing of pre-compiled commonly used queries\cite{MemSQL}.
	VoltDB tries to make its performance gains by what they call "Concurreny through
	scheduling"\cite{VoltDB}. This is the process of using a single-threaded pipeline 
	that performs the task it was scheduled. This limits the need for locking during
	transactions by intelligently scheduling the transactions so that locks are not
	necessary.
	\par\vspace{\baselineskip}
	OpenMemDB aims to take a different approach, one that is fully wait free. The goal is 
	to use powerful wait free data structures that will allow for a massively parallel 
	DBMS that can scale with the addition of processors and memory. It is our assumption 
	that the achievement of a fully weight free DBMS will achieve the performance gains 
	that have been lacking in the DBMS world while eliminating the complexity of 
	distributed systems, all while retaining full ACID compliance.
	
	\subsubsection{Other Database Management Systems}
	The two DBMS systems listed above are far from the only ones that exist. They were
	researched more heavily because of there perceived similarities to our project and
	because the documentation for them was thorough. However, there are numerous lesser
	known and less well documented DBMS that are worthy of note.
	\par\vspace{\baselineskip}
	Aerospike is an AGPL licensed NoSQL flash-optimized in memory 
	DBMS. Aerospike is an open source project that follows the similar pattern of 
	a distributed shared-nothing architecture that is appearing common among most
	in-memory DBMS. 
	\begin{quote}
	Aerospike architecture is derived from its core principles – NoSQL scalability and
	flexibility, along with traditional database consistency, reliability and extensibility.
	\cite{aerospike}
	\end{quote}
	\par\vspace{\baselineskip}
%%\includegraphics{graphics/RelatedWorksArchitectures/aerospike_dbms_architecture.png}
	\cite{aerospike}
	\par\vspace{\baselineskip}	
	Apache Geode is an Apache licensed distributed in-memory DBMS. Apache Geode is a brand
	new project and as such has very limited documentation. They claim on there github page
	to be 
	\begin{quote}
	... a data management platform that provides real-time, consistent access to 
	data-intensive applications throughout widely distributed cloud architectures.
	\end{quote}
	The specifics of their architecture is not listed. 
	\par\vspace{\baselineskip}
	dashDB is IBM's in-memory data warehouse. Self described as 
	\begin{quote}
	a high performance, massively scalable cloud data warehouse service, 
	fully managed by IBM. \cite{dashDB}
	\end{quote}
	dashDB comes in a couple different configurations, the one that most resembles 
	our project is the MPP, "Massively Parallel Processing", configuration. 
	MPP operates by allowing the data warehouse to leverage multiple servers 
	in a network cluster to process data simultaneously. 
	\par\vspace{\baselineskip}
%%	\includegraphics{graphics/RelatedWorksArchitectures/dashDB_mpp_architecture.png}
	\cite{dashDB}
	\par\vspace{\baslineskip}
	Each server in this data warehouse utilizes a number of key technologies, including: 
	\begin{itemize}
	  \item Dynamic in-memory processing: Even when a dataset
	  does not fit entirely in memory, dashDB still processes at
	  lightning fast speeds using a series of patented algorithms
	  that enable in-memory acceleration. While every workload
	  is different, dashDB only requires RAM size to be 5 percent
	  of the original pre-load source data size in order to run at
      in-memory optimized speeds.
     \item Actionable compression: dashDB performs a broad range
	  of operations—including joins and predicate evaluations—
	  directly on compressed data, therefore improving memory
	  and cache bandwidth, and saving CPU costs.
   	 \item Parallel vector processing: dashDB is CPU optimized
	  and designed for the latest generation of microprocessors.
	  Both multi-core parallelism and SIMD vector instructions
      enable dashDB to maximize hardware performance.
	  \item Data skipping: BLU enables dashDB to intelligently avoid
	  scanning entire ranges of column data that don’t qualify for
	  analysis, preserving time and resources.
	\end{itemize}
	
	dashDB uses a highly parallelized infrastructure optimized for columnar data
	exchange that is organized as such:
	\par\vspace{\baselineskip}
%%	\includegraphics{graphics/RelatedWorksArchitectues/dashDB_query_architecture.png}
	\cite{dashDB}
	\par\vspace{\baselineskip}
	
		
	
>>>>>>> dece7356307d5b3a7a613486f4c8c95eed3d4d7e
\newpage

\section{Design Summary}
\newpage

\section{Facilities and Equipment}
Our primary meeting place as a team was the computer science senior design lab in Harris Engineering
Center (HEC) 105. Workstations are provided along with collaboration equipment such as 
whiteboards and computers hooked up to TVs. A rack-mounted server was also made available and 
some members of the team were given virtual machines on it to use as development
platforms.
\par\vspace{\baselineskip}
Individual group members were able to use their personal machines as development platforms, though 
eventually all but the most trivial testing had to be moved to the server(s) we had access to
as their consumer-level hardware could not provide enough raw power to properly benchmark and
stress-test the system. The hardware used by each member:
\begin{itemize}
 \item Thor (Neil Moore's personal machine)
 \begin{itemize}
  \item{\makebox[4cm]{CPU:\hfill} AMD FX-8350 (8-core)}
  \item{\makebox[4cm]{RAM:\hfill} 16 GB DDR3}
  \item{\makebox[4cm]{Motherboard:\hfill} Asus M5A99X EVO R2.0}
  \item{\makebox[4cm]{GPU:\hfill} AMD Radeon HD 7870}
  \item{\makebox[4cm]{Primary Hard Disk:\hfill} PNY Optima 240 GB SSD}
  \item{\makebox[4cm]{Total Available Storage:\hfill} 3 terabytes}
  \item{\makebox[4cm]{Operating System:\hfill} Arch Linux}
 \end{itemize}
\end{itemize}
\par\vspace{\baselineskip}
In addition to the senior design lab, we also had access to the Scalable and Secure Systems lab server
thanks to Dr. Dechev. This server was the primary test-bed for OpenMemDB as consumer hardware simply
doesn't have the necessary parallization or memory needed in order to truly test and stress a
highly-parallelized and memory-intensive database management system. The server we were allowed to 
use had the following hardware characteristics:
\begin{itemize}
 \item 64 cores
 \item 312 gigabytes of RAM
 \item 1 terabyte of hard disk storage
\end{itemize}

\newpage

\subsection{Consultants, Subcontractors, and Suppliers}
Throughout the course of designing and developing this database management system we sought the advice
and knowledge of various people, as none of us are experts or even particularly knowledgeable in 
massively parallel systems. Our primary source of information and advice was Steven Feldman, the 
original developer of Tervel and its maintainer, even past his departure from the University of Central
Florida to Google. His insight into the internals of Tervel proved invaluable, as did his many 
explanations on the concepts of wait-freedom and how to create a wait-free system.
\par\vspace{\baselineskip}
Dr. Dechev also proved a reliable source to consult when either Steven was unavailable or when he
had particular insight into a problem we were facing. 
%Include stuff about the other guy...
\newpage

\section{Budget and Financing}
Thanks to the contributions from our sponsor and the nature of our project, we did not need to expend
any additional funds outside of normal maintenance each team member performed on their personal
machines.
\newpage

\section{Project Summary}
\newpage

\bibliography{final_document}
\bibliographystyle{acm}
\newpage

\appendix
\section{Appendix A}


\end{document}
