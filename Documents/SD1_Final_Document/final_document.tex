\documentclass[letterpaper, 12pt]{article}
%\documentclass[letterpaper,10pt]{scrartcl}

% TODO: Normilize our itemization template? (Periods at the end or not?)

\usepackage[utf8]{inputenc}
\usepackage[pass,letterpaper]{geometry}
\usepackage{graphicx}
\usepackage{listings}
\usepackage{color}
\usepackage[square,sort,comma,numbers]{natbib}
\usepackage{mathtools}

% Temporarily ignore graphics
% \renewcommand{\includegraphics}[2][]{\fbox{}}

\setlength{\parindent}{0cm}

\definecolor{codegreen}{rgb}{0,0.6,0}
\definecolor{codered}{rgb}{0.6,0,0}

\lstdefinestyle{omdb_cpp}{
keepspaces=true,
showspaces=false,
showstringspaces=false,
showtabs=false,
tabsize=2,
basicstyle=\footnotesize,
commentstyle=\color{codegreen},
keywordstyle=\color{blue},
stringstyle=\color{codered}
}
\lstset{style=omdb_cpp}

\title{Non-Blocking In-Memory Database\thanks{Sponsor: Dr. Damian Dechev}}
\author{Michael McGee, Robert Medina, Neil Moore, Jason Stavrinaky\\[1ex]
	\includegraphics{graphics/OMDB_logo_text_trans.png}\\[1ex]
}
\date{11/4/2015}

\pdfinfo{%
  /Title    (Non-blocking, In-memory Database)
  /Author   (Michael McGee, Robert Medina, Neil Moore, Jason Stavrinaky)
  /Creator  (Neil Moore)
  /Producer ()
  /Subject  (Final Document for COP4934)
  /Keywords ()
}

\begin{document}
\pagenumbering{gobble}
\maketitle
\newpage

\pagenumbering{roman}
\tableofcontents
\newpage

\pagenumbering{arabic}

\section{Executive Summary}
The database as a concept has always emphasized speed, as many other pieces of both software
and business rely on the information contained within them. As such, these pieces of software
are part of a class of software that strive for every piece of performance possible, either through
hardware improvements or algorithmic optimizations. While in-memory databases are not a new concept,
Random Access Memory (RAM) has, until recently, not been cheap enough to completely house a useful
set of data. The ability to do this in such a way beyond simply using main memory as a cache for the
actual data on the hard disk opens up the possibility of fully utilizing the processing power
available to the system fully.
\par\vspace{\baselineskip}
Utilizing the full hardware resources available to us requires the use of multiple threads or
multiple processes spread out over all the cores available. Accessing the same data location
from multiple threads or processes causes issues where threads are unable to progress further in their
work. This state is called deadlock, and is the bane of any programmer that creates multi-threaded
programs. Our project attempts to entirely avoid both deadlock and their common mitigator, locks,
by implementing and using wait-free data structures and algorithms.
\par\vspace{\baselineskip}
Wait-free is a guarantee that the system will always progress, as a whole, within a given period of time
regardless of work-load or contention over resources. The application of this guarantee to many
of the common algorithms and data structures familiar to computer scientists is still under heavy research
(as an example, there is still no widely-accepted implementation of a wait-free binary search tree).
\par\vspace{\baselineskip}
The objectives for this project is to successfully implement a SQL database that is both fully in-memory
and fully wait-free, even at the cost of performance. In order for this to be possible within the time
given to us, the scope of this project must be toned down and strictly enforced. While we will
attempt to be SQL-compliant, certain aspects of that standard heavily imply a type of mutual
exclusion be used which we obviously cannot use if we wish to remain wait-free. The pure size of
the standard is also a major obstacle to this objective.
\par\vspace{\baselineskip}
While our technical approach is prone to change as our understanding of the various concepts
and systems at play in a Database Management System (DBMS), we do have a general sense that we
wish to utilize a functional or imperative style of data flow. Objects will be used when appropriate
but we would not describe the overall design as object-oriented in any way. On the contrary, we would
find a much more suitable term in data-oriented programming as we are almost entirely dealing
with large amounts of arbitrary manipulation rather than the interaction of objects.
\par\vspace{\baselineskip}
To facilitate a wait-free system, we will implement a system where a pre-determined pool of threads
that will be assigned a queue of tasks, whether they be SQL queries or database-specific commands.
These threads will be almost entirely distinct and independent of the others, with as little
inter-communication as possible in order to avoid the possibility of deadlock or the necessity of
locking. The assignment of these tasks will be done by a work manager that will perform some form
of load-balancing when distributing the tasks among the pool of threads.
\par\vspace{\baselineskip}
Each thread will independently analyze, plan and execute its tasks in such a way that no shared memory
outside of the actual data is needed. The access to the data store, necessary in any database,
will be handled via a common set of interfaces that will then utilize the algorithms and data structures
given to us by our sponsor. The planning and optimization of these tasks will be primary source
of technical challenge in this project as the task language chosen (SQL) is declarative rather
than imperative or functional in nature. In other words, the tasks' language only tells us what
to retrieve rather than how to retrieve it.
\par\vspace{\baselineskip}
The advantages given to us by the approach detailed above is the inherent and explicit wait-freedomn
that occurs when the need for locking or shared state is removed entirely. As a result of that, we
hope to be at least competitive or comparable to current in-memory or traditional databases in terms
of performance.

\newpage

\section{Project Motivation}
As hardware reaches the limits in Moore's Law and processors stop becoming faster and faster
and instead focus on becoming more and more parallel, algorithms and the software that implement
them must adapt in order to maximize both performance and hardware utilization.
\par\vspace{\baselineskip}
% Graphic showing Moore's Law tapering off (graph of GHz and number of cores?)
%\includegraphics{graphics/Moores_Law_core_freq_comparison.png}
Recent research done by multiple universities and companies have yielded the concept of wait-freedom,
the guarantee that the entire system will make progress towards a goal in a given period of time.
This is a stronger guarantee than lock-freedom as lock-freedom only guarantees that a single thread
will make progress in a period of time. Our sponsor, Dr. Dechev, and his lab here at the University
of Central Florida have created a framework of wait-free data structures named Tervel ~\cite{tervel}.

\subsection{Personal Motivations}
\subsubsection{Neil Moore}
This project is one that I knew would be incredibly challenging, yet also had the greatest
potential in terms of personal growth and end product. These technologies are only going to
become more and more important and widespread throughout the software industry as hardware
continues to parallelize, memory becomes faster and cheaper, and wait-free/lock-free algorithms
become more mature. Experience working with high-performance, memory-intensive, and
standards-defined technologies opens the door for later opportunities in cutting-edge technologies.

\subsubsection{Michael McGee}
  My motivation for choosing this project is simply a desire to learn. I have always been
  interested in database systems as well concurrent programming. Nowhere do these issues
  more converge than in the project that Dr. Dechev proposed. I hope to gain from this
  project, a greater understanding of the underlying principles of database system design,
  as well as discover broader applications for lock-free and wait-free data structures. I
  know that throughout the course of this project I will learn a great deal and come out
  as a much improved software developer and computer scientist.

\subsubsection{Jason Stavrinaky}
I have always been interested in multiprocessor programming.  Considering the
future  of  programming  is  almost  certainly  heading  towards  a  multiprocessor
heavy paradigm, it is an essential skill.  Working on a big project involving this
is a great opportunity to learn.  Not only about multiprocessor programming
but also memory management.  In addition, creating an open source solution to
memSQL sounds like an exciting challenge to undertake while also contributing
to the programming community.

\newpage

\section{Broader Impacts}
An open-source, in-memory and wait-free database would allow for startups or students to
experiment with extremely high-frequency applications that still need relational
models without dealing with proprietary software such as MemSQL. As the product
will be wait-free, scalability with additional hardware should be near-linear as each
additional hardware-backed thread allows more concurrent SQL queries to be processed.
The applications of such a technology are the same as any other database system,
though the speed and scalability of the system lends itself to high-frequency and
relatively low storage size databases.
\par\vspace{\baselineskip}
In addition to the commercial or practical impacts, this project can also influence the
amount of research done by other companies or universities in this area, specifically the
usage of wait-free data structures. This can then lead to unique or more efficient
algorithms that then lead to better multi-threaded or multi-core programs. As most hardware,
even in the embedded sector, is moving to multi-core architectures these benefits could
then impact many areas within the technology sector. As a database that is quite decoupled
from its data structures this provides a unique test-case for stress-testing different wait-free
algorithms and data-structures given that they implement a base set of features.
\newpage

\section{Specification and Requirements}
The requirements for this project as given to us by our sponsor, Dr. Dechev, are as follows:
\begin{itemize}
 \item A wait-free, massively parallel, in-memory database
 \item Project must be open-source (MIT or BSD licensing)
 \item The database must be able to run a simple benchmark showing its performance characteristics against
 against both in-memory databases and conventional databases
 \item A workshop paper that details the problems solved by wait-freedom and how we
 implemented those solutions
 \item In-depth documentation on both the design and implementation of our database
\end{itemize}
\par\vspace{\baselineskip}
Dr. Dechev left the specifics of how we implement the database and interfaces to the database
up to us. In the interest of being similar to the database management systems we are being
tasked with competing with, we chose to implement the relational database model using SQL. This way we
can directly compare our feature level to other databases. This also allows us to easily generate
a shared test and benchmark suite to profile our project's performance characteristics as is needed
to satisfy by our requirements.
\par\vspace{\baselineskip}
As such, the following are the specification for this project:
\begin{itemize}
 \item A wait-free data store that can be performant when handling large amounts
 of memory without utilizing the hard disk
 \item A minimalistic implementation of the SQL standard:
 \begin{itemize}
  \item CREATE TABLE, DROP TABLE commands
  \item SELECT query, with qualifications such as lack of nested queries
  \item INSERT, UPDATE, and DELETE commands with certain clauses not implemented
  \item Certain datatypes:
    \begin{itemize}
      \item Integral types (e.g. INTEGER, SMALLINT, BIGINT)
      \item Boolean types (e.g. BOOLEAN)
      \item Character string types( e.g. CHAR,VARCHAR)
      \item Floating-precision types (e.g. FLOAT)
      \item Date and time types (e.g. DATE, TIME, TIMESTAMP)
    \end{itemize}
  \item Column aggregate functions: MIN, MAX, AVG, COUNT, SUM, etc.
  \item Table constraints: PRIMARY KEY, NOT NULL, DEFAULT, FOREIGN KEY, UNIQUE, AUTO\_INCREMENT
  \item WHERE, ORDER BY, GROUP BY, IN, BETWEEN clauses with support for boolean expressions within them
  \item ODBC connector implementation
 \end{itemize}
\end{itemize}

\newpage

\section{Research}

\subsection{IDEs}
% Author: Jason
When undertaking a big project, an often under-looked area is IDE research. IDEs provide
invaluable tools that aid the programmer in one convenient package. A few of these tools include:

\begin{itemize}
	\item Compilers
	\item Debuggers
	\item Version control integration
	\item Many, many more
\end{itemize}

As such, IDE choice is very important since they are essential to a stable, comfortable
work-flow. The following sections describe the IDEs we chose.

\subsubsection{Vim}

Vim stands for Vi-improved. Stemming from the popular Unix text editor Vi, Vim
provides an almost proper superset. Pretty much everything you can do in Vi you
can do in Vim. Additionally, Vim adds many features on top of Vi that caused Vim
to gain a huge following of programmers everywhere with plugin support being the main
reason. We chose Vim for that reason as well as its customizability. It can tweaked
to work exactly how we want it to work. With plugins, we can add essentially infinite
functionality. In the figure below, we have Vim running with the powerline plugin as
well as git integration.

%\begin{figure}
    %\centering
	%\includegraphics[scale=0.25]{graphics/vimide.png}
 %   \caption{Vim}
%\end{figure}

\begin{figure}
    \centering
	\includegraphics[scale=0.25]{vimide.png}
    \caption{Vim}
\end{figure}

\newpage

\subsubsection{Atom}
We chose Github Atom for similar reasons we did Vim. Atom is highly configurable, and it
comes with a built in package manager. Atom is basically a much more user friendly
version of Vim. When making quick edits, Atom does the job perfectly. Also, for
those of us who aren't too familiar with Vim.
%\begin{figure}
    %\centering
	%\includegraphics[scale=0.25]{graphics/atomide.png}
 %   \caption{Atom}
%\end{figure}

\begin{figure}
    \centering
	\includegraphics[scale=0.25]{atomide.png}
    \caption{Atom}
\end{figure}

\newpage

\subsubsection{Sublime Text}
Sublime text is a great, versatile text editor that also provides incredible customizability.
As competitor to Atom, Sublime Text has its advantages over Atom. Being older than Atom,
Sublime Text has many more packages available that gives it the edge in versatility. In
addition, Sublime has higher performance at the cost of CPU usage. While Atom uses less
CPU and more RAM. In all, both editors have their uses and are great for different things,
and as such we use them both.

%\begin{figure}
%    \centering
%	\includegraphics[scale=0.33]{graphics/sublimeide.png}
  %  \caption{Sublime Text}
%\end{figure}

\begin{figure}
    \centering
	%\includegraphics[scale=0.33]{sublimeide.png}
	\includegraphics[scale=0.25]{graphics/sublimeide.png}
    \caption{Sublime Text}
\end{figure}


\newpage

\subsubsection{Texmaker}
When writing research papers, there is nothing better than LaTex. However it can be really
inconvenient manually compiling and opening the pdf every time we want to see a change.
Texmaker takes care of that. It integrates all the LaTex packages we need into one piece
of software. It also has additional features such as preview windows, spell check,
structure browser, and more. It has become very useful to us for writing all the
papers for this project.

%\begin{figure}
    %\centering
	%\includegraphics[scale=0.25]{graphics/texstudioide.png}
 %   \caption{Texmaker/Texstudio}
%\end{figure}

\begin{figure}
    \centering
	\includegraphics[scale=0.25]{texstudioide.png}
    \caption{Texmaker/Texstudio}
\end{figure}
\newpage

\subsection{Other Tools}

\subsubsection{Zeal}
Undertaking a big project is a challenge. Not only is ample research required, but learning
new things is a big part of the process. This is not an easy task as it is a lot of information
to take in at once. Some say after learning a programming language, it is not hard to learn a
new one with similar syntax and the hardest part is just memorizing the small changes. This
is the case with us. Zeal is an offline documentation browser that lets us load in all
the docsets of almost any language and search through all of it in an instant. This tool
has become invaluable to our development process.

%\begin{figure}
%    \centering
%	\includegraphics[scale=0.25]{graphics/zealdoc.png}
 %   \caption{Zeal documentation browser}
%\end{figure}

\begin{figure}
    \centering
	\includegraphics[scale=0.25]{zealdoc.png}
    \caption{Zeal documentation browser}
\end{figure}

\newpage


\subsection{Database Management Systems}
To begin researching a database management system you must first understand what a
database is. According to the book "Database Management Systems" a database is

\begin{quote}
"... a collection of data, typically describing the activities of one or more related
organizations."\cite{ramakrishnan2000database}
\end{quote}

A database management system, according to the same text is

\begin{quote}
"... software designed to assist in maintaining and utilizing large collections of data".
\cite{ramakrishnan2000database}
\end{quote}

There are many different types of database management system that can achieve the
goal of maintaining large collections of data. Some are specialized for the larger, and
more volatile web data, while some are more suited for large persistent data.
\par\vspace{\baselineskip}
Some different types of database management systems are:
\begin{itemize}
\item Relational
\item Hierarchical
\item Network
\item Object-oriented
\item NoSQL
\item NewSQL
\end{itemize}

\subsubsection{Relational}
Perhaps the most common and well known of all database management systems is the
Relational DBMS. The relation data model is based off of tables that represent
the data, as well as the relationship among the data. It is defined in the book
"Fundamentals of Relational Database Management Systems" as such:
\begin{quote}
"The relational model uses a collection of tables to represent both data and
the relationships among those data. Tables are logical structures maintained
by the database manager. The relational model is a combination of three
components, such as Structural, Integrity, and Manipulative parts."
\cite{sumathi2007fundamentals}
\end{quote}
\begin{figure}
  \centering
  \textbf{Relational DBMS Structure}
  \includegraphics[scale=.5]{graphics/dbms_RDBMS_structure.png}
  \caption{Represents a disk based RDBMS}
\end{figure}
The three components can be further broken down.
\par\vspace{\baselineskip}
\textit{The Structural Part} is what defines the database as a collection of relations,
\par\vspace{\baselineskip}
\textit{The Integrity Part} is maintained using primary and foreign keys,
\par\vspace{\baselineskip}and
\textit{The Manipulative Part} are the tools, such as relational algebra and
relational calculus, that are used to manipulate the data.
\par\vspace{\baselineskip}
The author lists the key features of the relational model as follows:
\begin{itemize}
\item Each row in the table is called tuple
\item Each column in the table is called attribute.
\item The intersection of row with the column will have data value.
\item In relational model rows can be in any order.
\item In relational model attributes can be in any order.
\item By definition, all rows in a relation are distinct. No two rows can be exactly
the same.
\item Relations must have a key. Keys can be a set of attributes
\item For each column of a table there is a set of possible values called its
domain. The domain contains all possible values that can appear under
that column.
\item  Domain is the set of valid values for an attribute.
\item Degree of the relation is the number of attributes (columns) in the relation.
\item Cardinality of the relation is the number of tuples (rows) in the relation.
\end{itemize}

The idea of a \textit{key} is an important one in the world of relational database
management systems. So it is worth taking some time to talk about it. "A key is an
attribute or a group of attributes, which is used to identify a row in a relation."
\cite{sumathi2007fundamentals}
According to the author of "Fundamentals of Relational Database Management Systems" a
key can be classified into one of three categories.
\begin{itemize}
\setlength{\itemsep}{1pt}
\setlength{\parskip}{0pt}
\setlength{\parsep}{0pt}
\item Superkey
\item Candidate key
\item Primary key
\end{itemize}
A superkey is "a subset of attributes of an entity-set that uniquely identifies
the entities. Superkeys represent a constraint that prevents two entities from
ever having the same value for those attributes."\cite{sumathi2007fundamentals}
\par\vspace{\baselineskip}
A candidate key is "a minimal superkey. A candidate key for a relation schema
is a minimal set of attributes whose values uniquely identify tuples in the
corresponding relation."\cite{sumathi2007fundamentals}
\par\vspace{\baselineskip}
A primary key is a " designated candidate key. It is to be noted that the
primary key should not be null."\cite{sumathi2007fundamentals}
\par\vspace{\baselineskip}
There is also a key known as a foreign key, which is a "set of fields or attributes in one
relation that is used to “refer” to a tuple in another relation."
\cite{sumathi2007fundamentals}
\par\vspace{\baselineskip}
\par\vspace{\baselineskip}
There are many other aspects that full make up what it is to be a relational database
management system. The most important of which being relational algebra.
\par\vspace{\baselineskip}
"Relational algebra is a theoretical language with operations that work on
one or more relations to define another relation without changing the original
relation"\cite{sumathi2007fundamentals}
Operations of relational algebra include such things as selection operations, projection
operations, rename operations, union operations, intersection operations, difference
operations, division operations, as well as joins.
\par\vspace{\baselineskip}
The advantages of relational algebra is that is has a solid mathematical background.
This mathematical background is beneficial when it comes to optimization of queries,
because if two expressions can be proven to be equivalent a query optimizer can
substitute the more efficient operation whenever necessary.
\par\vspace{\baselineskip}
There are some disadvantages to using a relational model and to relational algebra in
particular. One disadvantage is the growing complexity of the data that needs to be
stored. Relational database management systems are good for linking together similar types
of data, and with the movement towards increasingly complex data-types these similarities
are becoming less common. Another disadvantage of an RDBMS is that it can be a complex
system to set up. A database administrator must take significantly more into consideration
when designing and RDBMS then when dealing with a simple object store.

\subsubsection{Hierarchical}
Hierarchical Databases are a data model in which the data is organized into a tree type
structure with a one-to-many relationship from the parent to the children. The data
within the tree are stored as records that are connected together through links. Each
record contains a set of fields with each field containing only one value.
\par\vspace{\baselineskip}
The hierarchical database model is typically used for large amounts of data that are
unlikely to change. It is recognized as the first database model used by IBM in
the 1960s.\cite{hierarchical_dbms_techopedia}
\par\vspace{\baselineskip}
\begin{figure}
  \centering
  \includegraphics[scale=.5]{graphics/dbms_RDBMS_structure.png}
  \caption{Example structure for hierarchical database}
\end{figure}

\subsubsection{Network}
The Network Model for a database management system is structures similarly to that of the
hierarchical model in that it consists of parent and child nodes. However the network
model does not limit itself to having every child have only one parent. In the network
model all children can have multiple parents, and obviously parents can have multiple
children. The network model also consists of items called "records" that which is a
collection of data items. Each data item has a name and a value.

\begin{figure}
  \centering
  \textbf{Network model record example}
  \includegraphics{graphics/dbms_network_model_record.png}
  \caption{Example of network model record}
  \cite{network_model_coronet}
\end{figure}

Each record can have it's own grouping within by grouping two or more elementary items.
A record can also contain a table, which is a collection of values that are grouped
under one data item.
\par\vspace{\baselineskip}
The data within the network model consists of two main parts: data objects, and
relationships.

\begin{figure}
  \centering
  \includegraphics{graphics/dbms_network_data.png}
  \caption{Network information model}
  \cite{network_model_coronet}
\end{figure}

Relationships between records can be implemented by using logical constructions.
This is called a "Data Set" . In the most basic case, each set consists of a father
and a child. The "Data Set" has the following properties:

\begin{itemize}
  \item Each set includes exactly one record of the first type. This record is called an Owner of the set.
  \item Each set may include 0 (i.e. an Empty set occurrence), 1 or N records of the same type. These records are called members of the data set.
  \item All members within one set occurrence have a fixed order (are sorted).
\end{itemize}

\begin{figure}
  \centering
  \includegraphics{graphics/dbms_network_model_relationship.png}
  \caption{Example of network model relationship}
\end{figure}

The basic setup of a network model is therefore a collection of record occurrences and
data sets.

\subsubsection{Object-oriented}
Object oriented databases are databases where the information stored within is organized
in the form of objects, like what would be used in an object oriented programming
language. An OODBMS combines the the features of an object-oriented language and a
DBMS. The OODBMS permits a much tighter coupling between the database and the application. This allows the programmer to maintain consistency within a single environment.
\par\vspace{\baselineskip}
In an OODBMS data is encapsulated in \textit{abstract data objects}, also known
as ADOs. ADOs all have the following properties:

\begin{itemize}
  \item It has a unique identity
  \item It has a private memory and a number of operations that can be applied to the current state of that memory.
  \item The values held in the private memory are themselves ADOs that are referenced from
within by means of variable identifiers called instance variables. Note the emphasis
“from within”, which underlines the idea of encapsulation, ie. such instance variables
or objects they denote or any organisation of the objects into any structure in the
private memory are not visible from outside the ADO.
  \item The only way that the internal state of an ADO can be accessed or modified from
outside is through the invocation of operations it provides. An operation can be
invoked by sending a message to it. The message must of course contain enough
information to decide which operation to invoke and provide also any input needed
by that operation. The object can respond to the message in a number of ways, but
typically by returning some (other) object back to the message sender and/or causing
some observable change (eg. in a graphical user interface).
\end{itemize} \cite{object_oriented_data_model}

\begin{figure}
  \centering
  \includegraphics[scale=.5]{graphics/dbms_oodbms_example.png}
  \caption{An example of an object-oriented model}
\end{figure}

\subsubsection{NoSQL}
NoSQL databases, which originally stood for "Non SQL" and now some people refer to as
"Not only SQL", are databases that store information in a non-relational, non-tabular
manner. Some of the ways that a NoSQL database management system may store information are:

\begin{itemize}
  \item Document-style stores
  \item Key-value stores.
\end{itemize}

Some examples of document-style stores are CouchDB and MongoDB. These are stores
"in which a database record consists of a collection of key-value pairs plus a payload."
\cite{stonebraker2010sql}
\par\vspace{\baselineskip}
Examples of key-value stores include MemcacheDB and Dynamo. These type of database
management system store data by strict key value pairs and are usually implemented by
distributed hash tables.
\par\vspace{\baselineskip}
With both of these styles you will be accessing the data one record at
at time as apposed to a SQL style database.
\par\vspace{\baselineskip}
There are typically two different arguments for choosing a NoSQL database management
system. These arguments can be summarized as such:

\begin{quote}
"There are two possible reasons
to move to either of these alternate
DBMS technologies: performance and
flexibility.
The performance argument goes
something like the following: I started
with MySQL for my data storage needs
and over time found performance to be
inadequate. My options were:

\begin{itemize}
  \item 1. “Shard” my data to partition it
across several sites, giving me a serious
headache managing distributed data
in my application or
  \item 2. Abandon MySQL and pay big licensing
fees for an enterprise SQL
DBMS or
\item Move to something other
than a SQL DBMS.
\end{itemize}\footnote{Formatting mine}

The flexibility argument goes something
like the following: My data does
not conform to a rigid relational schema.
Hence, I can’t be bound by the
structure of a RDBMS and need something
more flexible"
\end{quote} \cite{stonebraker2010sql}

\begin{figure}
  \centering
  \textbf{Example NoSQL Architecture}
  \includegraphics[scale=.4]{graphics/dbms_nosql_mongo_architecture.jpg}
  \caption{Example NoSQL Architecture}
  \citep{mongodb_architecture}
\end{figure}

\subsubsection{NewSQL}
NewSQL is a database management system that attempts to provide
the same scalable performance that a NoSQL DBMS can provide while still conforming to
the relational model. "Generally speaking, NewSQL data stores meet many of the
requirements for data management in cloud environments and also offer the benefits of
the well-known SQL standard"\cite{grolinger2013data}
\par\vspace{\baselineskip}
Since NewSQL is based on the relational model, every NewSQL DBMS must offer their clients
a pure relational view of data. This means that the data is interacted in terms of tables
and relations. This does not mean that the internal data representation is the same across
all NewSQL stores.
\par\vspace{\baselineskip}
"One of the main characteristics of the NoSQL and NewSQL data stores is their ability to
scale horizontally and effectively by adding more servers into the resource pool. Even
though there have been attempts to scale relational databases horizontally, on the contrary,
RDBs are designed to scale vertically by means of adding more power to a single existing
server"\cite{grolinger2013data}

\begin{figure}
  \centering
  \textbf{Partitioning, replication, consistency, and concurrency control capabilities}
  \includegraphics[scale=.4]{graphics/nosql_newsql_scalability_table.png}
  \caption{NoSQL and NewSQL Data Modeling}
\end{figure}

\newpage

\subsection{Work Manager}
The Work Manager is a conglomeration of multiple minor, but not insignificant, roles
within the DBMS. Specifically, it is the:
\begin{itemize}
  \item Main thread
  \item Sole communicator with the client
  \item Load-balancer between the primary worker threads
  \item System resource monitor
\end{itemize}

Work distribution between threads is done via a task-queue that each thread in the
thread pool pulls work from. Tasks are C++11 lambda functions that are packaged up
via the standard library's \lstinline|std::packaged_task<>|
 class.
This class provides an interface that allows us to generate a \lstinline|std::future<>|
object that is a thread-safe interface to a thread's return value. The bulk of the
ideas behind this thread pool were found online \cite{stackoverflow1} though we substituted
the command queue, that was initially implemented using a \lstinline|std::mutex|
and \lstinline|std::queue<>|, with a Tervel queue that allows
non-blocking and wait-free access and insertion into the command queue.
\par\vspace{\baselineskip}
The reference used mutual exclusion objects such as \lstinline|std::mutex|
which is contrary to our goal of a fully non-blocking and wait-free system. Using the Tervel
queue as detailed above allows us to remain non-blocking and wait-free as long as Tervel
is also considered non-blocking and wait-free. While the implementation uses condition variables,
they are used to sleep inactive threads rather than block waiting for a process. Letting these
threads sleep allow the operating system to schedule other threads in their place such as
background processes or other worker threads and their helper threads. This practice is also
considered common courtesy in multi-process systems so that the server is able to run more than
a single process and can conserve power in times of low demand.
\par\vspace{\baselineskip}
The task of communicating between the client and the server, which is the database itself,
is given to the work manager as it is the main thread and is where all the results of
the SQL query end up. This task also requires initialization and management of the network
sockets as well as the management of the socket identification numbers that are assigned
to specific connections. In turn, the module that manages these connections must be able to
determine which connection a query came from and where the query's results should be
returned to. The module that is in the best place for this work is the work manager
and has therefore been assigned these tasks.
\par\vspace{\baselineskip}
The work manager is also the part of the database management system that will be cognizant
of the amount of resources used by the system. Processor utilization, memory usage, thread
performance, and various other metrics listed below will be measured and able to be
requested by clients or management software. The full list of system resources measured
and reported by the server is as follows:
\begin{itemize}
 \item Shortest execution time of a query
 \item Longest execution time of a query
 \item Median execution time of a query
 \item Average ``load'' of a worker thread
\end{itemize}
\par\vspace{\baselineskip}

\subsection{Data Store}
%author Robert Medina
Data Store receives data requests and based on those requests will return a subset of data from
the database. This database will represent data in a table format using nested vectors. Each table
will be stored in a table hashmap to allow for easy table lookup. Each table contains a relational
schema for the type of data each column can accept. This is represented as a pair in the table
lookup hashmap.

%\includegraphics{graphics/Table_In_Memory.png}

Nested vectors are stored as row store architecture and column store architecture.

Row Store Architecture stores the memory of a row in a vector of multiple types as such
%\includegraphics{graphics/row_store_example.png}
Column Store Architecture stores the memory of a row in a vector of one type
%\includegraphics{graphics/column_store_example.png}

Side by Side Difference between row and column store
%\includegraphics{graphics/row_store_and_column_example.png}

Whenever a data request inquires about a table or more the data store will return a subset of that
data in the form of nested vectors.

%\includegraphics{graphics/row_store_and_column_example.png}

Row Store vs Column Store
Column Store:
The advantage of column based tables are faster data access, better compression, and better parallel processing.
In order to change a column from a row store architecture, each row in the table must be search in order to find the
column and then it requires to change the value. This makes it slow to check for integrity constraints. Column store
is a simply search through one vector row or column. Column vectors are very similar to each other value, which allows
for easy compression of data, unlike row vectors which could contain many different data types. In column store, data
is already partition into seperate columns so each column can be processed by itself.

Row Store:
The advantages of row based tables are easy row inserts into tables, and easy row access. This is useful when a query
wants to manipulate the entire row data instead of a few columns. In column store this would require multiple column vectors
being rewritten and maniuplated which would be a costly operation to perform.

%\includegraphics{graphics/query_of_table_example.png}

Seeing in the row store and column store figure previously, it would be
difficult to find the sales column in a row store column, and it is easily
noticeable to see how adding a new row in column store would require multiple
access to multiple column vectors.

\subsubsection{ACID}
%author Robert Medina

ACID:
Properties of a reliable database is outline as ACID. In order to guarantee that
database transactions are processed reliably a database must have atomicity, consistency,
isolation and durability.

Atomicity refers to the ability of a database to guarantee that all transactions
are performed or none of them are. Transactions that abort operations midway leaves
for data inconsistency. In the event that a transaction is aborted then the database
returns to the last commited transaction state.

Consistency refers to the ability of a transaction to take the database from one valid state
to another. Before a transaction and after a transaction the database must remain in a valid state.
Any data written to the database must be valid according to any constraints, cascades and triggers.

Isolation refers to the level of visibility of transactions to other users or systems. In other words,
transactions from two users cannot interact with one another and if two concurrent transactions occur then
those changes to the state of the database are not reflected to each other user. If those transactions occur
concurrently and rely on each other then locking will occur.

Durability means that once a transaction is commited then those changes on the database will be reflected to the user
and will be recorded pernamently. This is usually performed by backing up valid states of the database.

\subsubsection{Indexing}
%author Robert Medina
Indexing allows for fast retrieval from a collection of data. There are many ways to accomplish this,
and some ways are better suited depending on the constraints. Tree-based indexing and Hash-based indexing
are two popular solutions for an implementation of a database. Given a set of data in memory, indexing takes
a key data value and stores it into a data structure. Based off this key value, the indexing data structure
will point to the file in memory or memory address, depending on how the data is stored. Based on this, a
file scan of the database is now just reduced to scanning an index file.
%\includegraphics{graphics/indexentry.png}
However searching through an index file can still be a costly operation. Index files are smaller than the
data it is referencing but it can still use a considerable about memory space. Therefore it would be
reasonable to use a data structure for fast retrieval of data based on a range of values or based on
the actual data value stored.

\subsubsection{Tree-based Indexing}
%author Robert Medina
Tree-based indexing is an index file structured into a variant of a binary search tree. The tree is based off
the key value and references to where that key is stored. Indexing requires fast retrieval of data, low cost
insertion and low cost deletion of data. There are two types of trees that are useful for this type of operation,
ISAM and B+ Tree.

\subsubsection{B+ Tree}
%author Robert Medina
B+ Trees are a variant of a B Tree. B Trees is an n-ary tree that splits data
	based on a key value into nodes with T(j) subtrees and K(j) nodes, where j = (1<j<m).
	By definition a B tree is a tree with the following properties:
	Each node consists of T(i) subtrees and K(i) keys, where i = (0,...,j)
	There is a single root that contains a range of m children, 2<= m < j
	Each node contains a range of m children, (m/2) <= m < j
	For each sub-tree T(i) where i = (0,...,j) each sub-tree contains keys k(j) such that:
		T(0) sub-tree have keys k(0) where k(0) < k(i)
		T(i) sub-tree have keys k(i), i = (1, ..., j-1) where k(0) <= k(i) <= k(j)
		T(j) sub-tree have keys k(j) where k(j) < k(j)
	All T(i) sub-trees are either non-empty or empty.
	B Trees have a height of log(m)(N), m = number of children
%\includegraphics{graphics/B_Tree_Example.png}
B+ Tree differs slightly in the fact that B trees store their data (keys) within the internal nodes while
B+ trees only points to the data. All the data in a B+ is stored in the leaves of tree. In addition,
B+ Tree links all their leaves together with a doubly linked list.
%\includegraphics{graphics/B_B+_Tree_Difference.png}
The runtime for a B+ Tree is the following:
Searching is logm(n), m is index entries
Insertion is same as searching
Deletion is same as searching
Searching with range $( log_m n + k) $, m is index entries and k is number of data records
Key Compression:
The height of a B+ tree is log(m)(n), m = number of children, n = number of data entries.
Tradiationally for databases on disk the number of data entries would be based on the size of the data and
the size of the page. This is because tree nodes should fit on a single page, if B+ Tree leaves take up more
memory than a page offers then B+ tree is resized to fit onto a page. Since the page can only fit so much
data this is why the number of data entries is dependent on the size of data entries. So, it is reasonable
to max the number of data entries by making the size of data smaller. This enures that more entires will
fit on a page and the height of the B+ tree would remain small, thus keeping data retrevial performance fast.

\subsubsection{ISAM}
%author Robert Medina
ISAM:
Indexed Sequential Access Method is a variant of a B tree. In many ways it is similiar to a B+ tree however there are minor details about how memory is handled
that makes ISAM a better method than B+ tree. Although this is usually not the case. ISAM follows the exact same rules for a B tree but it does not contain
data internally with in its nodes. All the data from an ISAM resides in its leaf pages.

\begin{figure}
  \centering
  \textbf{INSERT TEXT}
  \includegraphics{graphics/ISAM)abstract_diagram.png}
  \caption{INSERT TEXT}
\end{figure}

All non-leaf pages contain pointers that reference to the data in a leaf page. It is common for
ISAM to never change the reference of a pointer since ISAM allocates memory statically. At creation
ISAM statically creates a default number of leaf pages. Insertions and deletions change the data
itself and not the leaf pages memory block. This static memory allocation in a sense can
provide a better performance than a dynamic B+ tree depending on the situation. At creation all
leaf pages are allocated squentially and sorted on the search key value. The non-leaf pages
are then allocated in memory. In the event that there is more data than ISAM can handle with its
default leaf page count then those pages go into an overflow page. Here is how the ISAM memory
allocation appears
\par\vspace{\baselineskip}

\begin{figure}
  \centering
  \textbf{INSERT TEXT}
  \includegraphics{graphics/ISAM_page_allocation.png}
  \caption{INSERT TEXT}
\end{figure}

Basic operations of insertion, deletion and search are very straightforward. An search starts at the
root node and determines which subtree to search by comparing the value of the key value in the node
with the search key value. This is standard in most tree data structures. Insertions and deletion are
based off the search operation and will either insert a new data to the pointer reference in the leaf
page or delete a data value from the reference of the pointer in the leaf page. None of the leaf
pages de-allocate memory. In addition the overflow pages are linked to the end of a leaf page array.
This way the overflow page will also be considered in the basic operations of ISAM.
\par\vspace{\baselineskip}

\begin{center}
\begin{tabular}{l | c}
  \hline
  Operation & Runtime \\ \hline \hline
  Search & $ log_f N $ \\ \hline
  Insert & $ log_f N $ \footnote{There is some overhead for leaf-page and overflow pages} \\ \hline
  Deletion & $ log_f N $\footnote{If an empty overflow exists it will be de-allocated} \\ \hline
\end{tabular}
\end{center}

\begin{figure}
  \centering
  \textbf{INSERT TEXT}
  \includegraphics{graphics/ISAM_example.png}
  \caption{INSERT TEXT}
\end{figure}

\begin{figure}
  \centering
  \textbf{After inserting 23*, 48*, 41*, 42*}
  \includegraphics{graphics/ISAM_example_insertion.png}
\end{figure}

\begin{figure}
  \centering
  \textbf{After deleting 42*, 51*, 97*}
  \includegraphics{graphics/ISAM_example_deletion.png}
\end{figure}

\par\vspace{\baselineskip}

ISAM in an indexing solution for the very expensive binary search on a file. This solves this
problem with consideration to the underlying memory devices in the system such as hard disks.

\subsubsection{Hash-based Indexing}
%author Robert Medina
Hash-based Indexing references a key value to a data entry using a hash function. This can be
used for indexing when there is an equal key value within the hash. Range operations using
a key value is not possible with a hash function since it would simply be too costly of an
operation. Hash-based indexing suffers from overflow chaining such as ISAM, which can hinder
performance. There exists multiple hash-based indexing such as Static Hashing, Extendible Hashing,
and Linear Hashing.
\subsection{Hash Function}
%author Robert Medina
Hash Function:
Good hash functions are deterministic, provide uniformity, and have variable range. Deterministic means
that for any given key value it will generate the corrseponding index consistently. This is
necessary to provide accurate data retrevial from a hash map. Uniformity means that a hash
map will evenly distribute hash index values over its value range. Without this the
performance will suffer greatly since the number of collisions will increase this hinders performance.
\subsubsection{Static Hashing}
%author Robert Medina
Static Hashing is a hash table based on a key value that maps to a bucket containing pages. These
data entries may be sorted but it depends on the application. This data structure is static
for the most part but it does allow for overflow page allocation. In the event of inserting
beyond the memory allocated, this data entry is placed into a new page and the page is added
to an overflow chain in the bucket.
\par\vspace{\baselineskip}
\begin{figure}
  \centering
  \includegraphics[scale=.75]{graphics/Static_Hashing_Abstract}
\end{figure}

A good hash function is imperative to uniformly distribute values over the collection of buckets.
An example of a good hash would be \[hash(key) = (a*key + b)\], a and b are constants. Some problems
of Static Hashing are the fact that it is static. When the index file is created bucket sizes
are known at time of creation, so pages can be stored successively in the buckets. However
as the index file continues to grow if the same key value is stored repeatedly then a long
overflow chain develops. Since the number of buckets are static if the index file shrinks
in size then there is wasted memory space. If the file grows too large then it results in
poor performance. Otherwise, the performance for operations is very fast. The following
is the runtimes:

\par\vspace{\baselineskip}

\begin{center}
\begin{tabular}{l | c | c}
  \hline
  Operation & Runtime & Number of I/O reads and writes \\ \hline \hline
  Search & O(1) & 1  \\ \hline
  Insert & O(1) & 2  \\ \hline
  Delete & O(1) & 2  \\ \hline
\end{tabular}
\end{center}

\par\vspace{\baselineskip}

Runtimes with Overflow Chains:
\par\vspace{\baselineskip}
\begin{center}
\begin{tabular}{l | c}
  \hline
  Operation & Runtime \\ \hline \hline
  Search & O(n) \\ \hline
  Insert & O(1) \\ \hline
  Delete & O(n) \\ \hline
\end{tabular}
\end{center}

If there are multiple collisions then the hashmap usefullness becomes impared. The hashmap devolves into a more
of a linked list so having long chains can really hinder performance as you see above. Therefore static hashing
is not a reliable method for inserting new objects since the possiblity of overflow chains and re-indexing can
ruin the property of a hashmap.
\par\vspace{\baselineskip}

Rehashing:
Intuitively a simple hash table with a pointer to a page would make sense. However in the case of
additional pages, without an overflow chain rehashing the table would be necessary. In this
case rehashing the table would be a costly operation. Including that the data structure is unusable
while rehashing is in progress. Dynamic hashing techniques solve this problem.
\par\vspace{\baselineskip}

Extendible Hashing:
Extendible Hashing is a hash map of pointers that maps to buckets with various levels of depth. This directory
is based on an index value using $ X\ modulo\ 2\textsuperscript{global depth} $. The global depth is the number of bits for the binary
representation of that index value. Each bucket in the directory has a shared data entry limit for the number of
entries each bucket can hold.

\begin{figure}
  \centering
  \textbf{Extendible Hashmap}
  \includegraphics{graphics/ExtendibleHashing_Depth_2}
  \caption{INSERT TEXT}
\end{figure}

\begin{figure}
  \centering
  \textbf{Spliting Occurs}
  \includegraphics{graphics/ExtendibleHashing_Spliting}
  \caption{INSERT TEXT}
\end{figure}


Whenever a data entry is inserted, the local depth of that bucket is compared with the global depth. If the local depth
is equal to the global depth then the directory is doubled and each pointer in the newly allocated memory is mapped to the
original buckets respectively. Also the global depth is increased by one and the directory index values are reindex to new
bit values. When this overflow occurs the maximum bucket gets split up into two buckets, one at the original index and the
second one at this new index value. This new index value holds the data entry and it's local depth is increased by one, along with
the original bucket local depth value.

\begin{figure}
  \textbf{Directory Doubling}
  \includegraphics{graphics/ExtendibleHashing_Depth_3.png}
  \caption{Directory doubles and the global depth increases to three.}
\end{figure}

\begin{center}
\begin{tabular}{l | c }
  \hline
  Operation & Runtime \\ \hline \hline
  Search & O(n) \\ \hline
  Insert & O(1) \\ \hline
  Delete & O(n) \\ \hline
\end{tabular}
\end{center}

In practice however, the length of a bucket is much smaller than static hashing and has
a uniform limit. This makes it more desirable in terms of performance on an average.

Linear Hashing:
%author Robert Medina

Linear Hashing is a dynamic hashing technique that adjusts to inserts and deletes in a stable manner.
Unlike Exttendible Hashing, Linear Hashing does not have a directory to where the data is
stored. Linear Hashing deals with collisions naturally and contains a lot of flexibility with
the timing of bucket splits. However if the data distribution is bery skewed overflow chains could cause Linear Hashing performance to be
worse than Extendible Hashing.
\par\vspace{\baselineskip}
Linear Hashing utilizes a famiy of hash functions H(a), h(1), h(2), ... , h(n), with each function having the property that each function's
range is twice of that of its predecessor. This means that if h(i) maps a data entry into one of M buckets then h(i+1) maps a data entry into one
of 2M buckets. This family typically contains a hash function $ h(i) = value = h(value) modulo (2^i N) $, where N is the initial number of buckets
which normally is 2. If N is a power of 2 then the hash function h will look at the last d(i) bits. d(o) is the number of bits needed to represent
N, and N can change as the number of buckets increase from splitting. d(i) is the next mod value, which is equal to d(i) = d(a)+i, for the next hash function
in this family of hash functions.
\par\vspace{\baselineskip}
In the split operation, there exists a split policy that determines when spliting is the appropiate. This can occur in terms of rounds. During a round, represented
as round number Level only hash functions h(level) and h(level+1) are in use. The buckets from first to last are split depending on the timing in a round. At any
time there exists buckets that splited, yet to be split and new buckets that were created recently.
\par\vspace{\baselineskip}
\begin{figure}
  \centering
  \textbf{INSERT TEXT}
  \includegraphics{graphics/Linear_hashing_splitting}
  \caption{INSERT TEXT}
\end{figure}

When a split occurs the data may not necessary reside in the newly split bucket. An overflow page is
added to store the newly inserted data entry. However this is not a big issue since bucket splitting
occurs in a round robin fashion, eventually all buckets are split including overflow chains.
This redistributes data into split buckets and each bucket is prevented from getting too large,
including overflow chains.
\par\vspace{\baselineskip}
The counter level is used to indicate the current round number. The bucket to be split is called
next and is initially the first bucket. The number of buckets is denoted by level and Nlevel.
N(o) is the number of buckets, also called N. A spliting policy can be a wide range of logic
such as whenever overflow chains are created or if a certain number of space is taken up with
in a bucket and so forth. Whenever a split is triggered by the split policy then the Next containing
the bucket will be split and the hash function h(level+1) will redistributes the entries
between the buckets to the newest created buckets. These definitions will be reflected in the
following figures below.

\begin{figure}
  \centering
  \textbf{Insertion of 43*, 43 mod (4) = 3 = (011)2 binary representation}
  \includegraphics{graphics/Linear_hashing_insertion}
  \caption{INSERT TEXT}
\end{figure}

\par\vspace{\baselineskip}
After the insertion, the bucket next is split into a new bucket. Each buckets between Next and Nlevel have not yet been split,
using h(level) on a data entry the result is the bucket location where the data will be stored.
When Next is equal to Nlevel - 1 and a split is triggered, the last of the buckets are split. The number of buckets after the split is twice the number at the beginning
of the round and a new round is started with a level incremented by 1 and next rest to 0. Since level is increased by one the number of potential keys to be hashed will
be doubled.
\begin{figure}
  \centering
  \textbf{INSERT TEXT}
  \includegraphics{graphics/Linear_hashing_level_increased}
  \caption{Notice that the number of bits increased as well, since d(i) = d(a) + i.}
\end{figure}

Linear Hashing compared to Extendible Hashing
Extendible Hashing and Linear Hashing are very much alike to each other in terms of their operations and how each data structure handle these operations.
When Linear Hashing splitting occurs in a round. The number of buckets to be split is up to the last bucket in the range N. This is essentially the same principle with
respect to Extendible Hashing but occurs more frequently. The hashing functions are very similar to Exttendible Hashing, in a sense that the hashing function relates to
a family of hashing functions. Linear Hashing with uniform distributions can have a lower average cost for equality selections since the directory level hiearachy is
eliminated but for skewed data distributions this can result in empty buckets which is is allocated in memory. This can lead to poor performance when compared to
Extendible Hashing.

\subsection{Data Compression}
%author Robert Medina
Data Compression:
	Lossless data compression is a type of data compression algorithm that allows
	the original data to be perfectly reconstructed from the compressed data. When
	compared to lossy compression, these algorithms are a type of classification that
	allows for an approximation of the orignal data to be recovered from compress data.
	Loseless data compression is useful for any type of application when memory space is
	necessary to be conserved. In the case of loseless data compression there are three
	popular algorithms, run-length encoding, huffman coding, and LZ77/78.
Run-Length encoding:
	Run-length encoding is a very simple form of loseless data compression that is sequences
	in which the same data value occur in many consecutive data elements are not stored as a
	single data value and counter instead of the duplicate form. In other words, this algorithm
	replaces large sequences of repeating data with only one item with a counter. When considering
	that an data primitive for ints are four bytes and a characater is 1 byte. It would be useful
	to represent multiple characters as a counter instead of multiple characters. The same principle
	continues for multiple integers and other data primitives.
	For example if a string line contain, "WWWWWWWWBWWWWWWWWB" then "8W1B8W" size would be considerably
	smaller than the first data input. In the case of the uncompressed data it took over 20 bytes to
	represent that input while run-length encoding took 6 bytes. This can be a huge impact on memory
	allocation in areas where memory is size sensitve such as main memory, cahce, or pages.
	This compression may not have the same impact in all applications since it requires data to be
	in a sequence and contain repeating data. In soe applications this may not occur as often as one would
	hope for but nonetheless it is a useful compression algorithm for repeating data sets.
Huffman coding:
	Huffman coding is a type of prefix code that is used for lossless data compression. Prefix code is a type
	of code system that is only distinguished by its possession of a prefix proferty. This prefix property requires
	that no code word in the system that is a prefix of any other code word in the system. Therefore each code word
	must be unique to their own scopes. The output from Huffman's algortihm is a variable length code table for encoding
	a symbol into a prefix.
	Huffman's Algorithm:
	Based on a given set of sybols with weights, Huffman's Algortihm determines the minimalistic required prefix coding
	table in order to represent the data set. This algorithm starts by creating a binary tree of nodes. These nodes
	are stored in a regular array which size depends on the symbol placed into the node. Initally all nodes are leaf
	nodes and all those nodes contain the symbol and the weight of the symbol. Internal nodes contain a symbol weight
	and links to two child nodes. Here is an example of a text that is initalized into Huffman's algorithm with the
	frequency of the characters being used as the weight of the symbol.
	"Eerie eyes seen near lake" translated into the following symbols "E e r i [space] y s n a r l k ."
	Those symbol weights are the following:
	\includegraphics{graphics/huffman_example_symbol_freq}
	Once the data set is prepared it is inserted into a priority queue. In this example the priority queue sets the least
	frequency symbol counter to be of the highest priority.
	\includegraphics{graphics/huffman_example_initial}
	The process begins with the leaf nodes turning into a new node with children whose symbol weight is smaller than
	the parent node. This is continued until all the nodes on the queue become attached to one big Huffman tree.
	Algorithmic Steps:
	Using a priority queue where the node with the lowest probability is given the highest priority.
	1.) Create a leaf node for each symbol and add it to the priority queue.
	2.) While there is more than one node in the queue
		3.) Remove the two nodes of the highest priority from the queue.
		4.) Create a new internal node with these two nodes as children and with the probability equal to
		the sum of the two nodes' probabilities.
		5.) Add the new node to the queue
	6.) The remaining node is the root node and the tree is complete.

	Continuing with the example, the first two leaf nodes are added as children to a parent node whose
	weight is the sum of the two children nodes
	\includegraphics{graphics/huffman_example_internal_node}
	\includegraphics{graphics/huffman_example_internal_node_added}

	This continues repeatedly until one tree is created from the data set.
	\includegraphics{graphics/huffman_example_end}

	The last node in the queue contains the root to the Huffman tree. This Huffman tree
	contains the new code words for each character. This new prefix code will allow for easy
	data representation for the text and obviously any data set with a symbol and an weight can
	be applied to this algorithm and can be compressed into a prefix code symbol table.
	Assuming that going left is 0 and going right is 1 on the tranversal of this tree, one complete
	word can be found after a tranversal of its symbol into the Huffman tree.
	\includegraphics{graphics/huffman_example_prefix_table}
	After the prefix table is applied to the original data set the following result is the compressed
	data for this data set.
	\includegraphics{graphics/huffman_example_compressed_result}
	Considering that the original size of the set in ASCII would take 8*26 = 208 bits, when compared to
	the newly compressed data the new compressed data size is 73 bits.

	The unique prefix property must be maintain in order for the Huffman property to remain a valid prefix
	data symbol table. No code is a prefix to any other code in a system as described initially.
LZ77/78:
	LZ77/78 are two lossless data compression algorithms. LZ77 algorithms achieves compression by replacing
	occurences of data with references to a single copy of that data existing earlier in the uncompressed
	data stream. Length-distance pair is a pair numbers that are used primarily for enconding matches. In order
	to spot matches the encoder tracks a small subset of data of the most recent data. This structure in which
	the most recent data is stored is called a sliding window. LZ77 also known as sliding window compression is
	a compression algorithm for sequential data compression. The sliding window consists of two parts, a search
	buffer and a look-ahead buffer. The search buffer contains a portion of recently encoded sequence data and
	the look-ahead buffer contains the next portion of data to be encoded.
	Encoding Algorithm:
	In order for the data in the look-ahead buffer to be encoded the encoder searches back through the
	search buffer until it finds a match to the first symbol in the look-ahead buffer. This distance between
	the pointer from look-ahead is called the offset. Once a match is found for the first symbol then the encoder
	determines whether or not the symbol matches with the symbol from the look-ahead buffer. The number of
	consecutive symbols in the search buffer that match consecutive symbols in the look-ahead buffer is called
	the length of the match. Now the encode encodes this longest match symbol into a type of format of triple
	variable pair, (offset, length of match, codeword for symbol).
	\includegraphics{graphics/LZ77_compression_code}
	The format is more clearly defined with in this example.
	\includegraphics{graphics/LZ77_example_window}

	Limitations of LZ77:
	A clear limitation of LZ77 algorithm is that the window only can allocate enough memory for the dictionary. At
	larger window sizes performance begins to dwindle. A smaller window size also means that valuable dictionary data
	encoding is being toss away because they are removed from the dictionary to make room for the newly encoded data sets
	from the look-ahead buffer.
LZ78:
	Much like LZ77 the LZ78 lossless data compression algorithm replicates many of the LZ77 compression features however
	LZ78 attempts to address some of the more underlying problems with LZ77 such as a finite amount of memory for
	the dictionary. With LZ78, the dictionary is an unlimited collection of seen phrases. LZ78 schemes work by inputing
	phrases into a dictionary and then when a repeated occurance of that phrase is found, the output of the format for
	the search buffer match will now be an index to the dictionary referencing to the search buffer format stored in
	the dictionary also known as phrase. LZ78 is only a slight modification of LZ77 and the compression algorithm is still
	the same format as LZ77 as well as the encoder.

\newpage

\subsection{Parallelism}
% Author: Jason
% TODO: restructure this section to flow better
With current strategies for processor research shifting paradigms from pure speed to cores,
we must also adjust to reflect the changes. Parallelism has become more and more essential
over the past few years to developing fast programs.

\par\vspace{\baselineskip}

It is important to take advantage of parallelization because processors have hit a wall of
reasonable clock speed\citep{processorspeed}. The maximum consumer processor clock speed has
hovered around 4GHz for many years now. Instead of increasing clock speed, processor
manufacturers are starting to introduce more cores and new technologies such as hyperthreading
to compensate for clock speed insufficiency. Processors just aren't getting faster fast enough.
Adopting parallelism strategies is the easiest way to take advantage of these new technologies.

\subsubsection{Code Speedup}
Parallelism has massive potential for increasing performance. Parallelization of algorithms
allows the processor to carry out many calculations simultaneously. In fact, if we can
parallelize the entire algorithm, a theoretical speed up of {\bfseries n} times can be
achieved where {\bfseries n} is the number of processors we can split the calculations on.
Most modern computers today have at least 2 cores, and 4 cores being common. This means
if we can parallelize every part of an algorithm, we can theoretically speed up performance
by 2 or 4 times for 2 or 4 cores, respectively.

\par\vspace{\baselineskip}
Of course, in reality the speed up is less than that. We have memory latency, IO, locks
(more on that in the next section), "inter-processor communication and coordination"\citep[p. 13]{artofmulti},
and much more to worry about. Still, the gain from parallelization is often large enough to spend
the time to implement. In order to decide whether or not it is worth it, we can make estimates of
how much of the code can be parallelized.

\par\vspace{\baselineskip}

\newpage
Consider the following situation:
\begin{quotation}
	"Five friends who decide to paint a five-room house. If all the rooms are the same size,
	then it makes sense to assign one friend to paint one room, and as long as everyone paints at
	about the same rate, we would get a five-fold speed-up over the single-painter case.
	The task becomes more complicated if the rooms are of different sizes. For example, if one
	room is twice the size of the others, then the five painters will not achieve a five-fold
	speedup because the overall completion time is dominated by the one room that takes the
	longest to paint"\citep[p. 13]{artofmulti}.
\end{quotation}

We can use {\bfseries Ahmdal's Law} to approximate the speedup to be gained from painting the house with five friends, or in a real scenario, the estimated parallelizable code.

\begin{equation}
	S = \frac{1}{1-p+\frac{p}{n}}
\end{equation}

Where:

\begin{itemize}
	\item {\bfseries S} is the potential speedup
	\item {\bfseries n} is the number of concurrent processors
	\item {\bfseries p} is the fraction of the code that can be executed in parallel
\end{itemize}

In our example, we have 5 rooms, one of them being worth 2. So in total we have 5 painters for 6 rooms, so
{\bfseries p} = \(\frac{5}{6}\).


\begin{equation}
	S = \frac{1}{1-p+\frac{p}{n}} = \frac{1}{1-\frac{5}{6}+\frac{\frac{5}{6}}{5}} = \frac{1}{\frac{1}{6}+\frac{1}{6}} = 3
\end{equation}

The potential speedup ends up being only 3 times faster with 5 painters. Even though a big section
(5/6 or roughly 83\%) could have been done in parallel. It is clear that parallelizing as much
code as is possible is very important to maintain high performance.

\subsubsection{Caveats}

Though parallelism is a great tool that certainly affects performance tremendously in a positive way,
it also has it's own caveats. Imagine we have two threads, {\bfseries t1} and {\bfseries t2}
working in parallel and that eventually there comes a point where both threads need to write a value {\bfseries i}.


\begin{table}[h]
\centering
\caption{Race Condition}
\begin{tabular}{|l|l|l|}
	\hline
	& {\bfseries Thread t1} & {\bfseries Thread t2} \\
	\hline
	1 & Read Value i & Read Value i \\
	\hline
	2 & Add 1 to i & Add 1 to i \\
	\hline
	3 & Write back to i & Write back to i \\
	\hline
\end{tabular}
\end{table}

Since execution of parallel threads does not guarantee that the above steps will occur in
any specific order, a few possibilities occur. If thread t2 executes step 1 while
t1 is any time between steps 1 and 3 or similarly if thread t1 executes step 1 while
t2 is between 1 and 3, we will get wrong values. In this case, the threads race
to get the value because no real order is established. Therefore, one of the
threads can get the old value before it is updated. This is called a {\bfseries race condition}.
Let's take a look at a specific example. Assume {\bfseries i} is set to 2,
then the following executes:

\begin{table}[h]
\centering
\caption{Race Condition}
\begin{tabular}{|l|l|l|}
	\hline
	& {\bfseries Thread t1} & {\bfseries Thread t2} \\
	\hline
	1 &  & Read Value i(2) \\
	\hline
	2 & Read Value i(2) & Add 1 to i \\
	\hline
	3 & Add 1 to i & Write back to i(3) \\
	\hline
	4 & Write back to i(3) &  \\
	\hline
\end{tabular}
\end{table}

Due to the unpredictable order of execution, we end up with {\bfseries i} = 3,
when in reality we should have gotten 4. How do we solve this problem?
We can use {\bfseries mutual exclusion}.

\par\vspace{\baselineskip}

{\bfseries Mutual exclusion} allows us to lock a piece of data to ensure no other thread can
access  or edit it while it is locked. This ensures that we don't read a value
before another thread writes to it, for example.

\par\vspace{\baselineskip}

Consider the updated scenario:

\begin{table}[h]
\centering
\caption{Race Condition}
\begin{tabular}{|l|l|l|}
	\hline
	& {\bfseries Thread t1} & {\bfseries Thread t2} \\
	\hline
	1 & Lock i & Lock i \\
	\hline
	2 & Read Value i & Read Value i\\
	\hline
	3 & Add 1 to i & Add 1 to i \\
	\hline
	4 & Write back to i & Write back to i \\
	\hline
	5 & Unlock i & Unlock i \\
	\hline
\end{tabular}
\end{table}

Locking the variable allows a thread exclusive access to it. In this case, it doesn't
matter if {\bfseries t1} or {\bfseries t2} gets it first, the thread that did
not get the lock will be locked out until the lock is released. Lets take a
look at our example with locks, assuming {\bfseries t1 got the lock first}:

\begin{table}[h]
\centering
\caption{Race Condition}
\begin{tabular}{|l|l|l|}
	\hline
	& {\bfseries Thread t1} & {\bfseries Thread t2} \\
	\hline
	1 & Lock i & See i is locked, wait \\
	\hline
	2 & Read Value i(2) & Wait\\
	\hline
	3 & Add 1 to i & Wait \\
	\hline
	4 & Write back to i(3) & Wait \\
	\hline
	5 & Unlock i & Wait \\
	\hline
	6 &  & Lock i \\
	\hline
	7 & & Read Value i(3)\\
	\hline
	8 & & Add 1 to i \\
	\hline
	9 & & Write back to i(4) \\
	\hline
	10 & & Unlock i\\
	\hline
\end{tabular}
\end{table}

In this case, {\bfseries i} correctly gets assigned 4.

\par\vspace{\baselineskip}

The caveat there is that this slows down parallel execution. Threads have to wait for
the variable to be unlocked, wasting valuable time. Of course, not all variables
will need locks, otherwise parallelism would not be any faster than serial code execution.

\par\vspace{\baselineskip}

Locks can be very useful tools to prevent race conditions, however they should be used carefully.
Using locks can cause a {\bfseries deadlock}. {\bfseries Deadlocks} can happen when a thread
{\bfseries t1} holds a lock and is waiting on a piece of data that is locked by another thread
{\bfseries t2}, while at the same time {\bfseries t2} is waiting on the lock from {\bfseries t1}.
This causes a loop, and neither lock ever gets released. The next figure illustrates this example.
%figure

\par\vspace{\baselineskip}

Not all threads are created equal. In reality some threads take up resources much more often than others.
"{\bfseries Starvation} describes a situation where a thread is unable to gain regular access
to shared resources and is unable to make progress. This happens when shared resources are
made unavailable for long periods by 'greedy' threads"\citep{oracleconcurrency}.

\par\vspace{\baselineskip}

%figure

In \textit{The Art of Multiprocessor Programming \citep{artofmulti} } it is stated that a good lock algorithm should have:

\begin{itemize}
	\item {\bfseries Mutual Exclusion}: Critical sections of different threads do not overlap.
	\item {\bfseries Freedom from Deadlock}: If some thread attempts to acquire the lock, them some thread will succeed in acquiring the lock.
	\item {\bfseries Freedom from Starvation}: Every thread that attempts to acquire a lock will eventually succeed.
\end{itemize}

These guidelines are great for safe parallel programming by today's standards, but we can do better.
We present our implementation of a DBMS without the use of locks, while still maintaining a highly parallelizable data structure core.

\par\vspace{\baselineskip}

It is important to understand that if we could remove locks, we would get a noticeable increase in performance.
{\bfseries Non-blocking} algorithms help us solve this problem. We will talk about them in the next section.

\subsection{Non-Blocking Properties}
% Author: Jason

Now that we have explained the traditional and most popular approaches to parallel programming, we introduce non-blocking algorithms.

\par\vspace{\baselineskip}
An algorithm that has the non-blocking property ensures the system makes progress. There are two types of non-blocking algorithms. These are defined as:
\begin{itemize}
	\item \textit{Lock-free}: Ensures at least one thread makes progress in a finite amount of time.
	\item \textit{Wait-free}: Ensures all threads make progress in a finite amount of time.
\end{itemize}
In order to take advantage of modern processors, we implement a wait-free data store module.
After some research, we have discovered that no current DBMS uses wait-free data
structures, at most they are lock-free. The core of OpenMemDB relies on non-blocking algorithms and data
structures. Non-blocking properties ensure a massively parallelizable database architecture.

\par\vspace{\baselineskip}

From looking at the definitions, it is clear that wait-freedom grants much more
powerful guarantees in regards to system progress. It is also easy to conclude that
it is much harder to implement. In the next few sections we discuss this, other parallel programming algorithm categories, as well as the advantages and disadvantages each of them.

We will talk about:


\begin{itemize}
	\item Blocking
	\begin{itemize}
		\item Starvation Free
	\end{itemize}
	\item Obstruction Free
	\item Lock Free
	\item Wait Free
	\begin{itemize}
		\item Bounded
		\item Population Oblivious
	\end{itemize}
\end{itemize}

\newpage

Before we continue, it's important to define {\bfseries fairness}. {\bfseries Fairness} is done to avoid starvation. It prevents one thread from holding a lock for too long, disallowing other threads to execute their {\bfseries critical section}(the part of their execution that cannot be done concurrently).
\subsubsection{Blocking}
A blocking algorithm is the most common of the list. Pretty much anything that uses locks is blocking. Consider the following example:

\begin{lstlisting}[language=Java]

	Lock lock = new Lock();

	public void method1(){

		lock.lock();
		dowork();
		lock.unlock();

	}

\end{lstlisting}


The code above simply creates a obtains a lock by calling lock.lock(). While it is locked, the thread will do some work with dowork() and when it's done it will unlock it via lock.unlock(). If for example, another thread wanted to acquire the lock, it would not be able to until method1() unlocks it. The thread would just wait until it can acquire the lock, otherwise doing nothing until then.

\par\vspace{\baselineskip}

{\bfseries Advantage}\newline
The advantage to this is that it is very easy to code. There is no worry about two threads accessing a piece of data at the same time, it just wouldn't be possible.
\par\vspace{\baselineskip}

{\bfseries Disadvantage}\newline
This is the slowest way of parallel programming. Whenever a lock is needed, one gives up almost all benefits of multiple threads.

\newpage
\subsubsection{Starvation Free}

\begin{quotation}
"As long as one thread is in the critical section, then some other thread that wants to enter in the critical section will eventually succeed (even if the thread in the critical section has halted)." \citep{artofmulti}.
\end{quotation}

An example of a Starvation Free algorithm would be some code with a very strict fairness policy. This allows all threads to be able to execute their critical sections.

{\bfseries Advantage}\newline
Allows all threads to have a change to execute its critical section.
\par\vspace{\baselineskip}

{\bfseries Disadvantage}\newline
Not much faster than Blocking, it just guarantees more threads will execute their critical section.

\subsubsection{Obstruction Free}

\begin{quotation}
"A synchronization technique is obstruction-free if it  guarantees progress for any thread that  eventually executes  in  isolation. Even  though other  threads  may  be  in the  midst  of  executing operations,  a  thread  is  considered to execute in isolation as long as the other threads do not take any steps." \citep{obsfree}
\end{quotation}

Obstruction freedom basically states that a threads makes progress only if it doesn't encounter any resistance from any other threads. In simpler terms: when a thread executes in isolation, it finishes in a finite number of steps\citep{artofmulti}.

\par\vspace{\baselineskip}
{\bfseries Advantage}\newline
Slightly more guarantees than Blocking.
\par\vspace{\baselineskip}

{\bfseries Disadvantage}\newline
Though it is stronger than Blocking, this is an even weaker guarantee than lock freedom.

\par\vspace{\baselineskip}
In all, Obstruction Free is a nice in-between solution

\subsubsection{Lock Free}
\begin{quotation}
	"A method is lock-free if it guarantees that infinitely often some method call finishes in a finite number of steps"\citep[p. 60]{artofmulti}.
\end{quotation}

Lock freedom simply says that at all times, at least one thread is making progress. It might take infinitely many steps, but it always makes progress. The chart on this page shows an easy to follow process to figuring out if an algorithm is lock free.

\begin{figure}
    \centering
	\includegraphics[scale=0.7]{lockfreechart.png}
    \caption{Is it lock free?}
\end{figure}
%\begin{figure}
%    \centering
%	\includegraphics[scale=0.7]{graphics/lockfreechart.png}

%\end{figure}

The following is an example of a lock free algorithm.

\begin{lstlisting}[language=Java]

	AtomicInteger atomicVar = new AtomicInteger(0);

	public void funcLockFree() {

        int localVar = atomicVar.get();

        while (!atomicVar.compareAndSet(localVar, localVar+1)) {
               localVar = atomicVar.get();
        }

}

\end{lstlisting} \citep{concurrencyfreaks}
In this case, we are incrementing an atomic integer in a loop. This guarantees that some thread is always working. If the compare and swap operation fails on some thread, that just means another thread is making progress and working on their compare and swap.

Lock free programming isn't easy, there are many problems that can be faced. Each problem requires a unique technique to solve. Some can't even be solved. The next figure shows a variety of these problems and some proposed solutions.

\begin{figure}[H]
    \centering
	\includegraphics[scale=0.7]{techniques.png}
    \caption{Techniques}
\end{figure}
%\begin{figure}
%    \centering
%	\includegraphics[scale=0.7]{graphics/techniques.png}

%\end{figure}

\newpage
\subsubsection{Wait Free}
\begin{quotation}
	"A method is wait-free if it guarantees that every call finishes its execution
	in a finite number of steps"\citep[p. 59]{artofmulti}.
\end{quotation}

\newpage

These algorithms form a hierarchy, a wait free population oblivious algorithm is also wait free bounded, and wait free, and lock free. The following diagram better illustrates this hierarchy.

\begin{figure}
    \centering
	\includegraphics[scale=0.25]{paralleltypes.png}

\end{figure}
%\begin{figure}
%    \centering
%	\includegraphics[scale=0.25]{graphics/paralleltypes.png}

%\end{figure}


As we can see, lock free is the most general and provides the least guarantee. Then wait free is next, then wait free bounded, and finally wait free population oblivious.
\newpage


\subsection{DataStore}
%author Robert Medina
data store design:
Write about boost
Introduction:
Boost variant library focuses on typesafe storage and retrieval of a bounded set of types that is
of discriminated unions. A variant data type exposes an interface independent of the current value's type.
The user define the set of types that are allowed in a variant and this can contain any number of disparate
variant instructions. Two ways to retrieve a type from the value that is held in a variant is either the program
must store this type or use boost visitor mechanism. This variant data type is stored as a stack instead of a heap
to avoid expensive memory allocations.
\par\vspace{\baselineskip}

BoundedType:
Every type that is specified as a template argument to variant must satisfy the above requirements. In addition certain
features of variant are available if its bounded types meet certain additional requirements. The minimun requirements for
a type to be accepted in a boost variant object are:
\begin{enumerate}
 \item CopyConstructible or MoveConstructible
 \item Destructor upholds the no-throw exception-safety guarantee
 \item Complete at the point of variant template instantiation
\end{enumerate}

Additional Features are:
\begin{enumerate}
 \item Assignable: variant is itself Assignable if and only if every one of its bounded
 types meets the requirements of the concept. (Note that top-level const-qualified
 types and reference types do not meet these requirements.)
 \item MoveAssignable: variant is itself MoveAssignable if and only if every one of its
 bounded types meets the requirements of the concept. (Note that top-level const-qualified
 types and reference types do not meet these requirements.)
 \item DefaultConstructible [20.1.4]: variant is itself DefaultConstructible if and only
 if its first bounded type (i.e., T1) meets the requirements of the concept.
 \item EqualityComparable: variant is itself EqualityComparable if and only if every one
 of its bounded types meets the requirements of the concept.
 \item LessThanComparable: variant is itself LessThanComparable if and only if every one
 of its bounded types meets the requirements of the concept.
 \item OutputStreamable: variant is itself OutputStreamable if and only if every one of
 its bounded types meets the requirements of the concept.
 \item Hashable: variant is itself Hashable if and only if every one of its bounded types
 meets the requirements of the concept.
\end{enumerate}

CopyConstructible is a type that can be constructed from a value or reference of the same
type. Move constructible type is a type that can be constructed from an rvalue reference of its type.
No-throw guarantee, also known as failure transparency are operations that are guaranteed to
succedd and satisfy all requirements even in bizarre situations. Should an exception
occur, it will be handled internally and not by the programmer.
\par\vspace{\baselineskip}

Using Boost variant:
Whenever a \lstinline|boost::variant| object is created the default constructor makes the first type its first bounded type.
A discriminated union container is defined by instantiating the \lstinline|boost::variant| with the desired argument types.
These types are called bounded types and derive from the BoundedType concept. In order to limit the different
types that bounded types can accept the \lstinline|BOOST_VARIANT_LIMIT_TYPES| must be manipulated to reflect this.
In order to retrieve data types from \lstinline|boost::variant| there are multiple functions available in order to accomplish this task.
Value retrieval can be accomplish by using the get function. Get takes in a type as an argument and returns the first
data value in a \lstinline|boost::variant| object that matches with that data type.
\par\vspace{\baselineskip}
\begin{figure}
  \centering
  \textbf{INSERT TEXT}
  \includegraphics{graphics/boost_variant_get_header.png}
  \caption{INSERT TEXT}
\end{figure}

While \lstinline|boost::get<T>| is an acceptable means for value retrieval it suffers from several flaws.
The get function can only work for one static type of \lstinline|boost::variant|,
the heterogenous data variant at declaration. If \lstinline|boost::variant| object
changes its data type then get will lead to logical errors that were not considered.
Thus boost variant needs a more robust un-static function to retrieve
data from a \lstinline|boost::variant| object. The variant library offers a compile-time visitation
via \lstinline|boost::apply_visitor|. Visitation requires the program to explicity handle or ignore bounded types.
\par\vspace{\baselineskip}

\begin{figure}
  \centering
  \textbf{INSERT TEXT}
  \includegraphics{graphics/boost_variant_visitation.png}
  \caption{INSERT TEXT}
\end{figure}

This provides the same functionality as get but now this code allows the program to write
generic visitors at compile-time. \lstinline|boost::apply_visitor| also contains one more benefit for
value retrieval. Delayed visitation is a type of \lstinline|boost::apply_visitor| that returns
a function object that operates on any variant given to it. This is useful when operating on a sequence of
variant types.
\par\vspace{\baselineskip}

Preprocessor macros:
The variant class greatly simplifies use for specific instructions of the template but it
significantly complicates use for generic instructions. When creating a variant instruction
it is very easy to declare an variant object but it is less clear on how such types are
accepted into a function. The Preprocessor library provides a powerful solution to this dilemma.
The Boost.Variant library provides a macro \lstinline|BOOST_VARIANT_ENUM_PARAMS| that
simplifies the process for the user of declaring variant types in function templates or
other things of such nature. It is possible to use a type sequence to specify bounded types
instead of using boost variant default function. Essentially a type sequence performs the
same functionality of Boost.Variant but if the program for whatever needs must edit and
create specific types of type sequences it is possible with \lstinline|make_variant_over|
function. As shown below \lstinline|boost::make_variant_over| creates a type sequence that
is customizable. Problems with the variant class template is parameter list is limiting in
two factors. There must be a implementation maximum limit of the types available and the
nature of parameter lists makes compile-time manipulation of the list excessively difficult.
This is where editing a type sequence on your own would be helpful in these two aspects
however not all compilers are able to this function.
\par\vspace{\baselineskip}

\begin{figure}
  \centering
  \textbf{INSERT TEXT}
  \includegraphics{graphics/boost_variant_type_sequence.png}
  \caption{INSERT_TEXT}
\end{figure}

Extending Visitation Even Further:
Visitation is a powerful mechanism for manipulating a boost variant object content. Binear visitation
extends this by allowing simultaneous visitation of content of two different variant objects.
\par\vspace{\baselineskip}

\begin{figure}
  \centering
  \textbf{INSERT TEXT}
  \includegraphics{graphics/boost_variant_binary_visitation}
  \caption{INSERT TEXT}
\end{figure}

Multi visitation:
Multi vistation is like binary visitation with respect to functionality except that it allows for
three or more boost variant objects to be compared. This maybe an useful tool when used for
multiple compares between variants objects with in tables.

\begin{figure}
  \centering
  \textbf{INSERT TEXT}
  \includegraphics{graphics/boost_variant_multi_visitation}
  \caption{INSERT TEXT}
\end{figure}

Boost variant over boost any:
Boost variant contains several advantages over boost any such as guaranting the type of its
content is a user-specificed set of types. Boost variant provides compile-time checked
visitation of its content, before run-time. This allows for safer type storage. Boost variant
enables generic visitation of its content and offers an efficient stack-based storage scheme
that avoids the overhead of dynamic allocation. Boost any had some advantages over variant such
as virtually any type for its template argument content, no-throw guarantee of exception safety
for swap operation and less expensive compile-time processor/memory demands. Overall boost
variant was choosen for its highly customable type storage. The user-specificed set of types
is something that was highly sought for our data structures.
\par\vspace{\baselineskip}

Write about types
	Data Store has multiple types predefined that allow for a variant of types that tables can be
	comprosided of. There are over seven data types that are used to make up data store. One of which
	uses data types that the query planner will used. This is how the query planner and the data store
	interact with one another.
1.) DataVariant made up of SQL data types with a boost variant vector
2.)	TervDataVariant is TervWrapper around an exisiting data variant.
3.)	DataTableRecord is a vector of a tervel data variant. This makes up one row in
	a table.
4.)	DataTable is a vector of a data table record. This makes up a table that contains multiple rows.
	This is how a table will be represented in data store. Each table will be registered into lookup table
	for data access.
5.)	TableSchema is a boost variant vector that stores the table schema
6.)	SchemaTablePair is a pair that bundles table nested vector with the table schema
7.)	TableMap is a hashmap that stores the location of schematablepair with the table name. This
	allows for easy fast instant access retrevial of tables.

Write about classes written
	Data Store is made up of one class called data_store that i responsible of retrieving
	data from the main memory.
write about functions
	Data Store is made up of multiple functions that create data tables and returns specific data.
1.) createRow:
		createRow takes in a DataVariant row and a tablename in order to find the specific table
		you want to insert the row into. The DataVariant row is checked by checkSchema function in
		order to see if the table schema registered into the lookup_table hashmap exists and whether
		or not the new row data type fits into the table schema.
2.) createTable
		createTable takes in a char array tablename and a TableSchema schema argument in order to add
		the string to the hashmap lookup_table as well as the table schema. This is added as a pair
		into the lookup table and now the table is registered into the lookup_table and memory is allocated
		for that table.
3.) checkSchema
		checkSchema takes in a char array tablename and checks whether or not if the table exists in the
		hashmap lookup_table. If it does exisit then it checks whether the TableSchema schema matches the
		schema from the lookup_table. If it does match up then it returns true otherwise false.
4.) getRow:
		getRow takes in a char array tablename and an int rowID in order to find the table in the lookup_table
		and finds the row based on the rowID index. It then returns the row pointer it found.
5.) getTable
		getTable takes in a char array tablename and finds the table from the lookup_table. If it finds the table
		then it returns the table pointer.

write about what is being passed in
*done so by diagrams and expanded by functions*

write about data structs
	Nested vectors are used to represent tables. Each table is stored into a hashmap called lookup_table to refer to
	tables. A pair is used to group up table schema boost variant vector and the data table itself. Nested vectors
	are used since it most is the most efficent way to represent memory. A hashmap is used to refer tables because each
	table contains a unique identifier char array. A hashmap provides instant access table lookup. A pair is used to group
	up the table with its schema. Each table have their own unique schema and it is retrievable from the table lookup hashmap.

write about performance of data structs
	Nested vectors are instant access O(1), if you have the index of the column or row. This makes data retrieval
	linear time O(n). In the case of a full table scan, it is only necessary to find the columns desired, then each other row
	column data can be retrieved by using the column index. If a full table scan with all columns are desired then this is O(n^2).
	A pair is fairly easy to retrieve data from since it is only two pointers or objects grouped together in an object. In the case
	for a hashmap, it provides instant access O(1), until the hashmap becomes more filled up and collisions being to happen. Normally hashmaps
	become reindex whenever they are around half full but it may be more advantageous to use dynamic hashing. This is something that will be determine based on performance testing on a training data set.

write about memory allocation (dynamic, not preprocessor)
1.) Memory management in C++
	Memory in a C++ program is divded into two parts, the stack and the heap. The stack contains all the variables
	declared inside the function and will take up memory from the stack. While the heap is responsible for all unused memory
	of the program. This memory can be allocated dynammically when the program runs but the program must also insure that this
	memory gets cleaned up after it is done with it.
	*diagram for stack and heap*
	Programs cannot predicte how much memory will be required at run time. This memory can be allocated by the new operator 			and is referenced by the address of the space allocated. This memory can also be deleted by the delete operator. This is used 			when memory is no longer used purposely. If it is not cleaned up then it creates garbage memory that could otheriwse be used 		resourcefully. There are other ways to dynammically allocate memory, such as malloc(), calloc(), and realloc(). There are major 		differences between the new operator and the other memory allocation functions.

	Delete and Free do not MIX
	The delete() function works for the new() function but it causes undefined behavior when used on a malloc(). If the free() function is used on a new() object then the same thing occurs. When deallocating memory for new or malloc it is necessary to apply the correct functions to free up memory. This can usually occur when user-defined functions are applied instead of the STL, standard template library for C++. This can vary from system to system but also from compiler to compiler. When the delete operation is called to delete the new object the compiler generates the following code.
	%includegraphics{graphics/new_delete_example}
	If the only function the programmer desire is to handle the raw uninitalized memory then new and delete operators should not be considered. Instead the operator new and operator delete should be used to allocate memory and deallocate. This is essentially the same as malloc and free.
	%includegraphics{graphics/new_operator_bypass}

	New and malloc mixed
	If a placement new is used to create an object in memory, not the new function but placement new used to store already allocated data from malloc() function then delete should not be used because like in the previous example, the delete function calls the operator delete to deallocate memory but the memory contained was not allocated by the operator new. This placement new just returns the pointer passed to it. So in general, if raw memory was passed to placement new and this memory was dynammically allocated then still memory must be deallocated to prevent a memory leak.
	%includegraphics{graphics/new_placement}

	New operator
	There is a difference between the new operator and operator new. The new operator does two things.
	First, it allocates enough memory to hold an object of the type inputed. Secondly, it calls a constructor to initialize an object of 		the type defined. This operator's behavior cannot be changed in ay way. The way the memory is allocated can be changed 			however. The new operator calls a function to perform the memory allocation. This can be overloaded to change the bahvior. 			The function at allocates memory for the new operator is "operator new".
	%includegraphics{graphics/operator_new_header}
	The operator new returns a void pointer to raw, uninitialized memory. The parameter, size_t, specifices how much memory to allocate. Whenever the programmer needs to call raw memory it is called by operator new like so.
	%includegraphics{graphics/operator_new_raw_memory}
	This line of code will return a raw chuck of meaningless memory.

	Placement new
	Sometimes there exists raw memory that was allocated dynammically but does not have a constructor to be linked up to. Invoking a constructor on an existing object is not logical. Constructors exist to initialize objects and objects can only be initialized once. Therefore there must be a way to call a constructor directly so it can be linked to already allocated memory. Placement new allows the programmer to directly work with a constructor. Here is an example of how placement new might be used.
	%includegraphics{graphics/placement_new_example}
	In the function constructWidgetInBuffer, the expressed return is a new operator that returns the location of the constructor by using an overloaded operator new function.
	%includegraphics{graphics/operator_new_overload}

	Error Handling
	By default if new() cannot find memory it calls a pointer to function called _new_handler()
	%includegraphics{graphics/operator_new_function}
	if _new_handler() can somehow supply memory for malloc() then everything is good otherwise an exception is thrown.

	New and malloc()
	The new operator creates an object that refers to the memory occupted. Malloc on the other hand will simply call a memory block with no object reference to the memory.

	Memory Corruption
		There exist two reasons that memory corruption could occur in a program. The source of the memory corruption and its manifestation may be far apart. This makes it difficult to finding the cause of the memory corruption in relation to its effect. The second reason is that symptoms occuring during bizzare conditions which make them significantly harder to replicate. Memory Corruption Errors can be categorized into four categories. One of the more common errors is using uninitialized memory. This refers to the contents of uninitialized memory are treated as garbage values. The usage of such garbage values can lead to unpredictable program behavior. Second error is using none-owned memory. It is common that pointers are used to access and modify memory. Whenever this pointer is a null pointer also known as a dangling pointer, or to a memory location outside of current stack or heap bounds, it is referring to memory that is not possessed by the program. Using these pointers causes serious flaws. Accessing such memory usually cause OS exceptions, that commonly leads to program crashes. Third, using memory beyond the memory that was allocated also known as a buffer overflow. For example, an array that is inside a loop contains a incorrect terminating condition then memory that is accessed beyond the bounds of the array may be manipulated that was otherwise not supposed to. Lastly, the four category that causes memory corruption is fauly heap memory management. Memory leaks and freeing non-heap or un-allocated memory are the most frequen erros caused by fault heap memory management. Since C++ gives so much freedom to allocate any amount of memory the program needs it can lead to such errors when the program does not clean yourself. More specifically memory leaks is allocated memory but are not released. This causes the program to consume memory and reduce the available memory in the heap.

/* cannot talk about such things, need research */
write about possible cache coherent
write about multiprocessor compliant
write an example of a table being stored into memory
write about the memory used to create a single table
write about data types
write about what it returns
write about query planner using logic instead of data store

\subsection{SQL Engine}
%author Mike McGee
One of the most important pieces of a Database Management System is the SQL Engine.
This component is responsible for receiving commands written in the standard query
language and transforming those commands into an internal representation that can be
executed by the DBMS. The SQL Engine often consists of three pieces: the tokenizer, the
parser, and the code generator.

\begin{figure}
  \centering
  \textbf{SQL Engine UML Activity Diagram}
  \includegraphics{graphics/SQLEngine_activity_overview.png}
  \caption{Diagram showing how the SQL engine handles and executes a SQL statement}
\end{figure}

\subsubsection{Tokenizer}
%initial author Neil Moore
To start the query planning and execution process, we must tokenize the provided query
from a straight string into a format readable by the parser. This tokenizer is very
similar to a generic compiler tokenizer though it's job is made considerably easier
by the importance of spaces in SQL queries/commands regarding token delineation.
\par\vspace{\baselineskip}
We encountered issues directly adapting current open-source implementations of a
SQL query/command tokenizer. As a tokenizer isn't a particuarly difficult piece of
code to replicate, we chose to implement our own tokenizer so that
we can optimize for our use cases without worrying about breaking things later on.
As most of the tokenizers were using C and we are using C++ we can also use C++-specific
constructs such as iterators and ``for each'' loops which introduce certain safety guarantees
that simply aren't possible with pointers as used in C.

\subsubsection{Parser}
%author Mike McGee
When researching SQL parsers we found that most database management systems use a
parser generator tool to develop a parser for the query language that they support.
The purpose of any parser generator is to implement a parser in the programming
language desired that will accurately parse the context free grammar that it
was passed in the grammar specification file.
\par\vspace{\baselineskip}
The two parser generators that we found were "Lemon" and "YACC". PostgreSQL uses
"YACC", which stands for "Yet Another Compiler Compiler", while SQLite uses
"Lemon", which was created to be a simpler alternative to YACC. Both tools will
generate a Look-Ahead Left-to-right Right-most-derivation (LALR) parser in C when
given a grammar specification file. Both tools share a base syntax in their grammar files,
but Lemon varies from YACC in some very important ways and enumerates those in its tutorial document.
\par\vspace{\baselineskip}
“It uses a different grammar syntax which is designed to reduce the number of coding
errors. Lemon also uses a more sophisticated parsing engine that is faster than yacc and
bison and which is both reentrant and thread-safe. Furthermore, Lemon implements features
that can be used to eliminate resource leaks, making is suitable for use in long-running
programs such as graphical user interfaces or embedded controllers.”\cite{lemon_parser}
\par\vspace{\baselineskip}
It is because of these benefits, especially thread safety, that we chose to use Lemon
as our parser generator. The Lemon parser generator is contained in one C code file and
is used by running the program with the grammar specification file as an argument.
This terminal command would resemble \textit{lemon gram.y}
Upon completion Lemon will produce between one and three files.\\ Those files are:
\begin{itemize}
	\item \textit{gram.c}: C code implementation of the parser
	\item \textit{gram.h}: A header file defining an integer ID for each terminal sybmol
	\item \textit{gram.out}: An information file that describes the states of the
	generated parser automaton
\end{itemize}
Lemon does not generate a complete program, it only creates a few subroutines that
implement a parser. It is up to the developer to call those subroutines in an appropriate
way in order to produce a complete system. In order to use a Lemon generated parser the
developer must first create the parser as follows:
\begin{lstlisting}
	void *pParser = ParseAlloc( malloc );
\end{lstlisting}
This call allocates and initializes a new parser and returns a pointer to it. The
parameter to the call is the subroutine used to allocate memory. For our purposes it will
most likely be something Tervel specific.

After the programmer is done using the parser they must free the memory that was allocated
to the parser using a subroutine of their choice. It is done as follows
\begin{lstlisting}
	ParseFree (pParser, free)
\end{lstlisting}
where free is the subroutine used to reclaim the memory, again probably Tervel specific for our purposes.

After the parser is allocated, the developer will provide the parser with a sequence of
tokens to be parsed. This is done by calling:
\begin{lstlisting}
	Parse(pParser, hTokenID, sTokenData, pArg);
\end{lstlisting}

\begin{quotation}
“The first argument to the Parse() routine is the pointer returned by ParseAlloc(). The
second argument is a small positive integer that tells the parse the type of the next
token in the data stream. There is one token type for each terminal symbol in the grammar.
The gram.h file generated by Lemon contains \#define statements that map symbolic terminal
symbol names into appropriate integer values. (A value of 0 for the second argument is a
special flag to the parser to indicate that the end of input has been reached.) The third
argument is the value of the given token. By default, the type of the third argument is
integer, but the grammar will usually redefine this type to be some kind of structure.
Typically the second argument will be a broad category of tokens such as ``identifier'' or
``number'' and the third argument will be the name of the identifier or the value of the
number. The Parse() function may have either three or four arguments, depending on the
grammar. If the grammar specification file request it, the Parse() function will have a
fourth parameter that can be of any type chosen by the programmer. The parser doesn't do
anything with this argument except to pass it through to action routines. This is a
convenient mechanism for passing state information down to the action routines without
having to use global variables.”\cite{lemon_parser}
\end{quotation}


\subsubsection{Query Planner}
The query planner is the most challenging and important piece of a typical DBMS as it
determines how the data in the SQL tables are retrieved which has a huge impact on
the overall performance of the DBMS. The need for a query planner comes from the
declarative nature of SQL queries, where the query doesn't tell you how to retrieve the
data but instead tells you what to retrieve. This is similar to languages such as Prolog where
a theorem solver is employed to determine how to solve the given problem. A large theorem
solver is impractical in a system with high-performance, low-latency requirements such as
a DBMS, so we must generate our own significantly stripped-down version that can handle a
specific set of query relations extremely fast.
\par\vspace{\baselineskip}
Luckily, due to the subset of features we are supporting in this project, we do not need to
implement a very strong or robust query planner with large amounts of optimizing or scheduling
capability. This is due to the fact that we have determined that the queries we are supporting
can be decomposed into a series of predicates that then can be evaluated almost independently.
These predicates are modeled in a way similar to the tree formed by boolean expressions, as shown
in the graphic below.
\par\vspace{\baselineskip}
\begin{figure}
 \centering
 \textbf{A Boolean Expression as a Binary Tree}
 \includegraphics{graphics/boolean_expression_tree.png}
 \caption{A typical boolean expression found within SQL decomposed into a binary tree}
\end{figure}


Therefore, we deduced that there are three different types of predicates that a given tree of
boolean expressions can be reduced to. They are as follows:
\begin{enumerate}
 \item Value predicates
 \item Column predicates
 \item Nested predicates
\end{enumerate}
\par\vspace{\baselineskip}
The value predicate is the core predicate when determining which rows to return to the client.
It performs a boolean operation on either two literal values or a column and a literal value.
This is the general predicate that the other two specialize upon and is by far the most important
in terms of implementing the WHERE clause as defined by the SQL standard.
\par\vspace{\baselineskip}
The column predicate is the predicate that is a true specialization of the value predicate that is
needed to handle an edge case, namely the case when a query is comparing two different columns.
This comparison requires extra information that cannot be contained within a normal value predicate
so that a cross-reference between the two tables can be generated. Whether a row satisfies this
predicate is determined by whether or not the row exists within the cross-reference contained by
the predicate. This predicate is particularly important as it implements a very foundational
part of the relational model, the ability to point to a specific row or range of
rows between tables (i.e. foreign keys).
\par\vspace{\baselineskip}
The nested predicate permits a boolean operation to be performed upon the results
of two predicates that are considered its children in the tree of expressions to be
evaluated. This predicate has a more limited subset of operators compared to
the value and column predicates and is limited to the following operators:
\begin{itemize}
 \item AND
 \item OR
 \item XOR
\end{itemize}
This predicate enables complex boolean expressions to be evaluated as part of queries or
commands which then allows users to perform powerful and specific queries upon the database.
\par\vspace{\baselineskip}
As an example of the reduction of a query into a series of predicates, take a
typical statement that would be supported by OpenMemDB:
\begin{lstlisting}[language=SQL]
SELECT A.*
FROM A,B
WHERE A.x = B.x AND
      (B.y = 1 OR B.z = 2);
\end{lstlisting}
This query references two tables, A and B, and features a three-level boolean expression.
Using the model defined previously, we are left with five predicates:
\begin{enumerate}
 \item An overarching nested predicate using the AND operator between its children, 2 and 3, that are defined below.
 \item A column predicate checking the equality between A.x and B.x.
 \item Another nested predicate that uses the OR operator between its children, 4 and 5, that are defined below.
 \item A value predicate checking the equality between B.y and the literal value ``1''.
 \item A value predicate checking the equality between B.z and the literal value ``2''.
\end{enumerate}
This model lends itself particularly well to a top-down evaluation model of this binary tree
where a particularly large or complex subtree can then be given to a worker thread to be evaluated.
Special care has be taken when handling truly large trees of predicates so that the entire
pool of worker threads is not exhausted by a single statement, but that is easily handled
by a hard limit of workers per statement.


\subsection{Related Works}
Over the last 30 years the there has been tremendous advancements in computing
hardware. "Processors are thousands of times faster and memory is thousands of
times times larger"\cite{stonebraker2007end}. Most technologies have advanced
along with hardware, however database management systems have struggled to improve
at a similar rate. This is mostly due to concurrency issues. "Existing studies show
that current database engines can spend more than 30\% of time in
synchronization-related operations (e.g.locking and latching), even when only a
single client thread is running."\cite{soares2015database}
\par\vspace{\baselineskip}
There have been several attempts to solve this problem. Some of which will sacrifice
some data consistency in order to achieve better performance. Still others
remain fully ACID compliant and attempt to parallelize individual steps in the
DBMS or solely use multiple threads when executing query plans. Then there are those
that attempt to implement some level of lock freedom into their DBMS.
\par\vspace{\baselineskip}
\subsubsection{MemSQL and VoltDB}
MemSQL\footnote{MemSQL can be configured as a Columnstore that stores data on disk}
and VoltDB are both fully in memory DBMS as is OpenMemDB. This is where the
similarities end as far as OpenMemDB is concerned. MemSQL and VoltDB on the other
hand both use distributed systems to achieve performance gains. MemSQL differs from
VoltDB in a few ways, the most important being it's use of lock free data structures
for storing data and its storing of pre-compiled commonly used queries\cite{MemSQL}.
\begin{figure}
  \centering
  \textbf{MemSQL Two-tiered Architecture}
  \includegraphics[scale=.5]{graphics/memsql_dbms_architecture.png}
  \caption{MemSQL Architecture}
\end{figure}
VoltDB tries to make its performance gains by what they call "Concurreny through
scheduling"\cite{VoltDB}. This is the process of using a single-threaded pipeline
that performs the task it was scheduled. This limits the need for locking during
transactions by intelligently scheduling the transactions so that locks are not
necessary.
\begin{figure}
  \centering
  \textbf{VoltDB Serialized Architecture}
  \includegraphics[scale=.5]{graphics/voltdb_serialized_architecture.png}
  \caption{VoltDB Serialized Processing}
\end{figure}
\par\vspace{\baselineskip}
OpenMemDB aims to take a different approach, one that is fully wait free. The goal is
to use powerful wait free data structures that will allow for a massively parallel
DBMS that can scale with the addition of processors and memory. It is our assumption
that the achievement of a fully weight free DBMS will achieve the performance gains
that have been lacking in the DBMS world while eliminating the complexity of
distributed systems, all while retaining full ACID compliance.

\subsubsection{Other Database Management Systems}
The two DBMS systems listed above are far from the only ones that exist. They were
researched more heavily because of there perceived similarities to our project and
because the documentation for them was thorough. However, there are numerous lesser
known and less well documented DBMS that are worthy of note.

\begin{itemize}
  \par\vspace{\baselineskip}
  \item Aerospike is an AGPL licensed NoSQL flash-optimized in memory
  DBMS. Aerospike is an open source project that follows the similar pattern of
  a distributed shared-nothing architecture that is appearing common among most
  in-memory DBMS.
  \begin{quote}
  "Aerospike architecture is derived from its core principles – NoSQL scalability and
  flexibility, along with traditional database consistency, reliability and
  extensibility."
  \cite{aerospike}
  \end{quote}
  \par\vspace{\baselineskip}
  \begin{figure}
    \centering
    \includegraphics[scale=.25]{graphics/aerospike_dbms_architecture.png}
    \caption{Aerospike Architecture}
  \end{figure}
  \cite{aerospike}
  \par\vspace{\baselineskip}
  \item Apache Geode is an Apache licensed distributed in-memory DBMS. Apache Geode is
  a brand new project and as such has very limited documentation. They claim on there
  github page to be
  \begin{quote}
  "... a data management platform that provides real-time, consistent access to
  data-intensive applications throughout widely distributed cloud architectures."
  \cite{aerospike}
  \end{quote}
  The specifics of their architecture is not listed.
  \par\vspace{\baselineskip}
  \item dashDB is IBM's in-memory data warehouse. Self described as
  \begin{quote}
  "a high performance, massively scalable cloud data warehouse service,
  fully managed by IBM." \cite{dashDB}
  \end{quote}
  dashDB comes in a couple different configurations, the one that most resembles
  our project is the MPP, "Massively Parallel Processing", configuration.
  MPP operates by allowing the data warehouse to leverage multiple servers
  in a network cluster to process data simultaneously.
  \par\vspace{\baselineskip}
\includegraphics[scale=.45]{graphics/dashDB_mpp_architecture.png}
  \cite{dashDB}
  \par\vspace{\baselineskip}
  Each server in this data warehouse utilizes a number of key technologies, including:
  \begin{itemize}
    \item Dynamic in-memory processing: Even when a dataset
    does not fit entirely in memory, dashDB still processes at
    lightning fast speeds using a series of patented algorithms
    that enable in-memory acceleration. While every workload
    is different, dashDB only requires RAM size to be 5 percent
    of the original pre-load source data size in order to run at
in-memory optimized speeds.
\item Actionable compression: dashDB performs a broad range
    of operations—including joins and predicate evaluations—
    directly on compressed data, therefore improving memory
    and cache bandwidth, and saving CPU costs.
   \item Parallel vector processing: dashDB is CPU optimized
   and designed for the latest generation of microprocessors.
   Both multi-core parallelism and SIMD vector instructions
nable dashDB to maximize hardware performance.
   \item Data skipping: BLU enables dashDB to intelligently avoid
   scanning entire ranges of column data that don’t qualify for
   analysis, preserving time and resources.
 \end{itemize}

  dashDB uses a highly parallelized infrastructure optimized for columnar data
  exchange that is organized as such:
  \par\vspace{\baselineskip}
\includegraphics[scale=.45]{graphics/dashDB_query_architecture.png}
\cite{dashDB}
\par\vspace{\baselineskip}
\item eXtremeDB is the in-memory variant of the McObject family of data management projects.
  It stores its data entirely in main memory, eliminating the need for disk I/O.
  eXtremeDB claims to have an "Ultra-small" footprint stating that through
  streamlining of core database functions they are able to reduce their RAM footprint
  to around 100KB. They also do not translate the data they store in memory. They
  store the data exactly how it will be used by the application. "No mapping a C data
  element to a relational representation"\cite{extremeDB} eXtremeDB claims to fully
  support ACID properties. It does this by ensuring that operations grouped into
  transactions will complete together or the database will be rolled back to a
  pre-transaction state.\cite{extremeDB}
  \par\vspace{\baselineskip}
\includegraphics[scale=.5]{graphics/extremeDB_dbms_architecture.jpg}
  \cite{extremeDB}
  \par\vspace{\baselineskip}

  \item GemFire is a distributed, in-memory, shared-nothing, NoSQL key-value store.
  GemFire is designed for working with operational data needed by real time
  applications. It is not meant for working on very large quantities. In order to
  achieve massive speeds GemFire relies on being primarily, not fully,
  main memory based. " It uses highly-concurrent main-memory data structures to avoid
  lock contention and a data distribution
  layer that avoids redundant message copying, and it uses native serialization and
  smart buffering to ensure messages move from node to node faster than what
  traditional messaging would provide."\cite{gemfire}
  \par\vspace{\baselineskip}
  \includegraphics[scale=.5]{graphics/gemfire_peer_architecture.png}
  \par\vspace{\baselineskip}

  \item Hekaton is Microsoft's database engine that is optimized for memory resident
  OLTP data. Hekaton is fully implemented within SQL Server and can be utilized from
  within.
  \begin{quote}
  "Hekaton is designed for high levels of concurrency but does not
  rely on partitioning to achieve this. Any thread can access any row
  in a table without acquiring latches or locks. The engine uses latchfree
  (lock-free) data structures to avoid physical interference
  among threads and a new optimistic, multiversion concurrency
  control technique to avoid interference among transactions"\cite{hekaton}
  \end{quote}
  Hekaton stores it's data in two different formats: a lock-free hash table and a
  lock free B-tree called a Bw-tree. Data is always accessed with an index lookup.
  \par\vspace{\baselineskip}
  Hekaton's Architecture:
\begin{itemize}
  \item The Hekaton storage engine manages user data and indexes.
  It provides transactional operations on tables of records, hash
  and range indexes on the tables, and base mechanisms for storage,
  checkpointing, recovery and high-availability.
  \item The Hekaton compiler takes an abstract tree representation of
  a T-SQL stored procedure, including the queries within it, plus
  table and index metadata and compiles the procedure into native
  code designed to execute against tables and indexes managed
  by the Hekaton storage engine.
  \item The Hekaton runtime system is a relatively small component
  that provides integration with SQL Server resources and
  serves as a common library of additional functionality needed
  by compiled stored procedures.
  \item Metadata: Metadata about Hekaton tables, indexes, etc. is
  stored in the regular SQL Server catalog. Users view and manage
  them using exactly the same tools as regular tables and indexes.
  \item Query optimization: Queries embedded in compiled stored
  procedures are optimized using the regular SQL Server optimizer.
  The Hekaton compiler compiles the query plan into native
  code.
  \item Query interop: Hekaton provides operators for accessing data
  in Hekaton tables that can be used in interpreted SQL Server
  query plans. There is also an operator for inserting, deleting,
  and updating data in Hekaton tables.
  \item Transactions: A regular SQL Server transaction can access
  and update data both in regular tables and Hekaton tables.
  Commits and aborts are fully coordinated across the two engines.
  \item High availability: Hekaton is integrated with AlwaysOn,
  SQL Server’s high availability feature. Hekaton tables in a database
  fail over in the same way as other tables and are also
  readable on secondary servers.
  \item Storage, log: Hekaton logs its updates to the regular SQL
  Server transaction log. It uses SQL Server file streams for storing
  checkpoints. Hekaton tables are recovered when a database is recovered.
  \end{itemize} \cite{hekaton}

  \item SAP HANA
  is short for "High Performance Analytic Appliance" and is an in-memory,
  column-oriented, relational database management system. SAP HANA employs a
  massively parallel in-memory architecture in order to eliminate the I/O bottleneck
  that slows disk backed database management systems.
  \par\vspace{\baselineskip}
  \includegraphics[scale=.5]{graphics/sap_hana_dbms_architecture.png}
  \cite{saphana}
  \par\vspace{\baselineskip}
\end{itemize}

\newpage

\section{Design Summary}
%% A summary describing the overall architecture should go here. Followed by subsections
%% for each major piece.
  \subsection{libomdb: The Connection API}
  With any database management system it is common to provide an easy way to connect
  to a specific database, and to execute commands that act on that database. The most common
  way to provide this connectivity is to support either ODBC or JDBC. However, the work
  required to be an ODBC or JDBC compliant database is considerable, and given that it is
  not one of the requirements of our project we will not be going down that road. We have
  decided in stead to design our own connection library, which we are calling
  "libombd". libomdb will provide all of the absolutely necessary components needed to
  connect to a specific database and to execute all of the supported SQL operations on
  that database.
  \par\vspace{\baselineskip}
  libomdb is broken in to two main operations: connecting to a database, and performing
  operations on the database that you are connected to. In order to connect to the database
  the application developer will call the
  \lstinline[basicstyle=\ttfamily]|connect()| method that is provided by the libomdb API.
  This method will return a \lstinline[basicstyle=\ttfamily]|Connection| object that can
  then be used to execute queries or commands to the database that was connected to.
  \par\vspace{\baselineskip}

  \begin{figure}
    \centering
    \textbf{Overall libomdb Class Architecture}
    \includegraphics[scale=.5]{graphics/libomdb_Class.png}
    \caption{libomdb Class Diagram}
  \end{figure}

  In order to connect to a database the application developer must provide the
  host-name of the server, the port the server is listening on, and the name of the
  database that is being connected to. These are provided as arguments to the
  \lstinline[basicstyle=\ttfamily]|connect(host-name, port-number, database-name)|
  method.
  \par\vspace{\baselineskip}
  The connection object returned to the application developer after making this method
  call represents a connection to only the database that was passed in the call. If the
  application developer needs a connection to another database, they must create a new
  connection object through a separate
  \lstinline[basicstyle=\ttfamily]|connect()| call.
  \par\vspace{\baselineskip}
  Once the connection is established, the returned connection object is then used for all
  operations on the database. Queries can be made against the database by calling
  the \lstinline[basicstyle=\ttfamily]|executeQuery( query )|
  method. This will return a \lstinline[basicstyle=\ttfamily]|Result| object, which can
  then be used to access the results of the query. Accessing the results of the query
  through a \lstinline[basicstyle=\ttfamily]|Result| object is simple and can be done by
  by cycling through the rows in the
  \lstinline[basicstyle=\ttfamily]|Result| and accessing the data at the column that
  is desired. Contained in the \lstinline[basicstyle=\ttfamily]|Result| object is a
  \lstinline[basicstyle=\ttfamily]|ResultMetaData| object than can be used to obtain
  information about the \lstinline[basicstyle=\ttfamily]|Result| that was returned.
  The \lstinline[basicstyle=\ttfamily]|ResultMetaData| object corresponding to the
  \lstinline[basicstyle=\ttfamily]|Result| object is obtained by calling the
  \lstinline[basicstyle=\ttfamily]|getMetaData()| method and contains
  methods for accessing the number of columns in the
  \lstinline[basicstyle=\ttfamily]|Result| through the
  \lstinline[basicstyle=\ttfamily]|getColumnCount()| method, the
  label of a specific column through the
  \lstinline[basicstyle=\ttfamily]|getColumnLabel( index )|,
  and the data type of a specific column with the
  \lstinline[basicstyle=\ttfamily]|getColumnType( index )| method.
  \par\vspace{\baselineskip}
  A sample of how to query for data and cycle through the results is:

  \begin{lstlisting}[language=C++]
Connection conn = Connection.connect("localhost", 3310, "users");

Result result = conn.executeQuery("SELECT * FROM Employees;");

ResultMetaData metaData = result.getMetaData();

while (result.next()) {
  // Access data
  for (int i = 0, j = metaData.getColumnCount(); i < j; ++i) {
    cout << result.getValue(i);
  }
}
  \end{lstlisting}
  \par\vspace{\baselineskip}

  \begin{figure}
    \centering
    \textbf{Sequence Diagram for Querying Data}
    \includegraphics[scale=.5]{graphics/libomdb_query_activity.png}
    \caption{Querying Database Sequence Diagram}
  \end{figure}
  \par\vspace{\baselineskip}

  An update command can be executed in much the same way as a query. First
  a connection needs to be established to the desired database using the
  \lstinline|Connection.connect()| method. Then the desired non-query
  command is executed using the \lstinline|executeCommand( command )| method.
  A \lstinline|CommandResult| will be returned from this call,
  which is a struct that has two fields: a boolean field,
  \lstinline|isSuccess|, and an integer field,
  \lstinline|numAffected|, which represents the number of rows
  that were effected by the command.
  \par\vspace{\baselineskip}
  An example of a command executed against a database would be something like this:
  \begin{lstlisting}[language=C++]
Connection conn = Connection.connect("localhost", 3310, "users");

CommandResult result;
result = conn.executeCommand("UPDATE employees SET pay=pay+1;");

if (result.isSuccess) {
  cout << "Rows affected: " << result.numAffected;
}
  \end{lstlisting}

  \begin{figure}
    \centering
    \textbf{Sequence Diagram for Executing Command}
    \includegraphics[scale=.5]{graphics/libomdb_command_activity.png}
    \caption{Executing Command Sequence Diagram}
  \end{figure}

  \par\vspace{\baselineskip}

  \subsubsection{Design}
  There are two main classes used in the libomdb library:  \\
  \lstinline|Connection| and \lstinline|Result|. The uses of these classes
  are obvious. \lstinline|Connection| is used to represent a connection to
  a specific database and \lstinline|Result| is used to hold the results of a
  query against the connected database.

  \par\vspace{\baselineskip}
  The \lstinline|Connection| class is made up of two private
  member fields \lstinline|m_socket_fd| and
  \lstinline|m_metaData|.
  \lstinline|m_socket_fd| is the the socket file descriptor that is
  used to communicate with the server hosting the database.
  \lstinline|m_metaData| is the related
  \lstinline|ConnectionMetaData| object that contains some
  information about the connection represented by the
  \lstinline|Connection| object. Some of the information contained
  in this \lstinline|ConnectionMetaData| object are
  the name of the database that the \lstinline|Connection| object
  is connected to, and whether or not the connection is still active. The
  \lstinline|Connection| class also contains three public
  instance methods and two public static methods. The instance methods are:
  \begin{itemize}
    \item \lstinline|executeCommand( string command ): CommandResult|
    \item \lstinline|executeQuery( string query ): Result|
    \item \lstinline|getMetaData(): ConnectionMetaData|
  \end{itemize}
  The static methods contained in the \lstinline|Connection| class
  are:
  \begin{itemize}
  	\item \lstinline|Connection.connect(string hostname, uint16_t port, string db): Connection|
  	\item \lstinline|Connection.disconnect(Connection connection)|
  \end{itemize}
  As has been seen a number of times the \lstinline|connect()|
  static method is used  to establish a connection to a database, and the
  \lstinline|disconnect()| method is used to terminate a connection.
  \lstinline|executeCommand()| in reality just sends a string of
  text to across the socket connection represented by the
  \lstinline|Connection| object's
  \lstinline|m_socket_fd| to the database server where it will be
  parsed and executed. The \lstinline|executeQuery()| command
  does nearly the exact same thing but will receive a different data structure from the
  database in return.

  \begin{figure}
    \centering
    \textbf{Connection Class Layout}
    %\includegraphics{graphics/libomdb_connection_layout.png}
	\caption{Connection layout}
  \end{figure}
  \par\vspace{\baselineskip}

  The \lstinline[basicstyle=\ttfamily]|Result| class is made up of two private members:
  \begin{itemize}
    \item \lstinline|m_rows|
    \item \lstinline|m_metaData|
  \end{itemize}
  \lstinline|m_rows| is a
  \lstinline|vector<ResultRow>|,
  \lstinline|ResultRow| is a typedef for
  a \lstinline|vector<DataValue>|, and
  \lstinline|DataValue| is a typedef for
  \lstinline|boost::variant<>|.
  So with all typedefs removed \lstinline|m_rows| is
  actually a
  \lstinline|vector<vector<boost:variant<> > >|.

  \lstinline|m_metaData| is the
  \lstinline|ResultMetaData| object that describes the
  \lstinline|Result| object. The contents of the
  \lstinline|ResultMetaData| class will be covered later.
  \par\vspace{\baselineskip}
  The \lstinline[basicstyle=\ttfamily]|Result| class also contains thee instance methods:
  \begin{itemize}
    \item \lstinline|getMetaData(): ResultMetaData|
    \item \lstinline|getValue(int index): DataValue|
    \item \lstinline|next(): bool|
  \end{itemize}
  \lstinline|getMetaData()| is used to obtain the meta-data
  information about the \lstinline[basicstyle=\ttfamily]|Result|.
  \lstinline|getValue( index )| will return the
  \lstinline|DataValue| of the column at the specified index.
  \lstinline|next()| moves the
  \lstinline|Result| forward one row, if there are still rows
  remaining in the \lstinline[basicstyle=\ttfamily]|Result|. If there is not, it does
  nothing and returns \lstinline[basicstyle=\ttfamily]|false|.

  \begin{figure}
    \centering
    \textbf{Result Class Layout}
    %\includegraphics{graphics/libomdb_result_layout.png}
    \caption{Result Layout}
  \end{figure}
  \par\vspace{\baselineskip}
  A quick overview of the the \lstinline|ResultMetaData| class
  is necessary to fully understand the how the
  \lstinline|Result| class is meant to work.
  The \lstinline|ResultMetaData| class contains only one
  private member, which is
  \lstinline|m_data|, a
  \lstinline|vector<MetaDataColumn>|.
  \lstinline|MetaDataColumn| is a struct that contains two fields:
  a \lstinline|string name| and a
  \lstinline|SQL_TYPE type|. \lstinline|name|
  is the name of the column and \lstinline|type| is the
  \lstinline|SQL_TYPE| of the column, i.e. "VARCHAR", "DATE", etc.
  \par\vspace{\baselineskip}
  The \lstinline|ResultMetaData| class also has
  three instance methods available:
  \begin{itemize}
    \item \lstinline|getColumnCount(): uint32_t|
    \item \lstinline|getColumnLabel(int index): string|
    \item \lstinline|getColumnType(int index): SQL_TYPE|
  \end{itemize}

  \lstinline|getColumnCount()| returns the number of
  columns contained in the \lstinline[basicstyle=\ttfamily]|Result|.
  \lstinline|getColumnLabel(index)| returns the label of the
  column at the passed in index, and
  \lstinline|getColumnType(index)| returns the type of data that
  is stored in the column specified by index.

  \begin{figure}
    \centering
    \textbf{ResultMetaData Class Layout}
    %\includegraphics{graphics/libomdb_resultMetaData_layout.png}
    \caption{ResultMetaData Layout}
  \end{figure}

  %% Starting here, talk about the use of boost::variant.
  %% Reasons for using it, how it is used, how to get value back from it.
  \newpage
  \subsection{SQL Engine}
  The core of any Database Management System(DBMS), particularly one that implements the SQL standard, is the
  engine that drives the actual parsing, planning, and execution of SQL queries and commands.
  The module in our DBMS that handles those tasks is the SQL engine and it is specially designed to
  be thread-safe, performant, and wait-free. Wait-freedom is the core requirement of the project
  and it supersedes all other requirements, so it is imperative that this module is wait-free. Keeping
  that requirement in mind, we utilized C++11 features and single-responsibility modeling to
  ensure that the various threads do not interact in ways that would require synchronization.
  \par\vspace{\baselineskip}
  One of those ways is utilizing a feature in the C++ memory model that was introduced in C++11, specifically
  the \lstinline|thread_local| storage specifier. That specifier instructs the compiler
  to create a copy of each variable for each individual thread, and ties that variable's lifetime
  to the lifetime of the thread. While not used extensively, this feature removes the guesswork
  of whether a variable is safe to be made \lstinline|static| or whether a copy
  would need to be made in each thread's initialization routine via dynamic memory (e.g.
  \lstinline|new| or \lstinline|malloc|).
  \par\vspace{\baselineskip}
  \begin{figure}
   \centering
   % C++ memory model diagram
    \textbf{C++ Memory Model}
    \includegraphics{graphics/cpp_memory_model.png}
    \caption{Diagram of the C++11 memory model as used by OpenMemDB}
  \end{figure}
  \par\vspace{\baselineskip}
  Another decision in the design of the SQL engine that was important to maintaining wait-freedom and
  thread-safety was the adoption of the Lemon parser engine. Lemon was built to be thread-safe and
  reentrant without locks, meaning that each thread essentially has its own instance of the parser
  and is independent of other threads. This cleanly avoids the need for costly, and forbidden
  in our case, locks and other synchronization primitives. In addition, we have complete control over
  the grammar from which Lemon generates its parser which gives us opportunity to ensure there
  is no lack of progress made within the parsing of a query due to a poorly-constructed or ambiguous
  grammar.
  \par\vspace{\baselineskip}
  The above decisions help guarantee wait-freedom within the parsing stage of the SQL engine, but there
  is still the task of the planning and execution of SQL queries and commands. The strategies employed in
  planning the execution of these queries has been detailed earlier in this document, so we will not
  revisit them in detail. As a refresher, we have devised a way to reduce a series of boolean expressions
  into a tree of predicates that a table's records can then be evaluated against. In order for a record
  to be in the result set of a query, it must satisfy the entire tree of predicates. Due to our reduced scope
  when implementing the SQL standard, we can comfortably guarantee that there will always be some form
  of progress when devising the execution plan. This is mainly due to the fact that advanced joins and
  other SQL constructs like nested queries, that OpenMemDB will not be supporting, require a full-featured
  solver that may not be able to progress in a way that can be called wait-free.
  \par\vspace{\baselineskip}
  \begin{figure}
   \centering
   % SQL Engine generation of predicate tree activity diagram
   \textbf{Predicate Tree Generation}
   \includegraphics{graphics/sqlengine_predicate.png}
   \caption{Diagram of the process used to generate the tree of predicates
	    used within the SQL Engine component of OpenMemDB}
  \end{figure}
  \par\vspace{\baselineskip}
  Execution of the query plan is also within the purview of the SQL engine and also must be guaranteed to be
  wait-free as well as performant. As our use case puts us entirely within memory, we have the necessarily low
  latencies to feel comfortable in performing full table scans in the most naive case with options to
  use indices and caches when available or reasonable. This will make us, in theory, faster than our disk-based
  counterparts such as PostgreSQL or MySQL. This task within the SQL engine is the one that will actually
  interact with the data store and actually performs operations upon the data stored there. We rely on the Tervel
  data structures to guarantee us wait-freedom as we access the data, but otherwise we can verify that
  progress in evaluating the query will be made in such a way to satisfy our wait-freedom requirement.
  \par\vspace{\baselineskip}

  \subsection{Work Manager}
  The Work Manager, as the module responsible for distributing the system's workload across the
  pool of worker threads, wears many small but important hats. It is the entry point of the
  process and functions as the main thread that spawns all other threads in the system. The Work
  Manager also is the owner of the various TCP sockets and other system resources that OpenMemDB
  uses to communicate with clients. Communication between threads is also handled via the
  Work Manager using Tervel's queue data structure and various C++11 constructs. The need for this
  jack-of-all-trades module is that many of these tasks cannot be easily split out into
  worker threads like the SQL Engine can and a couple act as a centralizing force, such as the
  thread pool management and work distribution. Sharing sockets and connections across thread
  boundaries also introduces complexity and the need for synchronization between threads which
  can interfere with our core requirement to be non-blocking and wait-free.
  \par\vspace{\baselineskip}
  The specific task of managing network connections between OpenMemDB and its clients is a fairly
  straightforward one as we don't have complex server-client interactions and relatively tame
  throughput expectations. Linux in particular exposes a well-documented and mature C interface
  to setup and use network or UNIX sockets which favors using C-style freestanding functions
  rather than a more object-oriented approach, at least in the initial setup and initialization
  of the socket. As such, we only created a class for the connection between the server and client
  and is suitably named \lstinline[basicstyle=\ttfamily]|omdb::Connection|. This class holds
  information about the client as well as the socket file descriptor that is needed to send
  or receive data over the socket. Instances of the class are created when the server detects an
  incoming new connection and they persist until the client disconnects or network conditions
  prohibit further communication such as the Internet going down.
  \par\vspace{\baselineskip}
  The freestanding functions mentioned above abstract away the process of listening to a port
  and accepting a connection. They are defined as:
  \begin{lstlisting}[language=C++]
omdb::NetworkStatus omdb::ListenToPort(uint16_t port_id,
				       uint32_t* socket_fd,
				       bool dont_block = true)

omdb::NetworkStatus omdb::AcceptConnection(uint32_t socket_fd,
					   uint32_t* conn_fd,
					   sockaddr_storage* conn_addr)
  \end{lstlisting}
  The \lstinline[basicstyle=\ttfamily]|omdb::NetworkStatus| datatype is used to return various
  possible error conditions and is internally used to classify whether the error is fatal
  or can be recovered from. Due to our goal of being non-blocking, we utilize various specially-defined
  functions in the Linux network API such as \lstinline[basicstyle=\ttfamily]|accept4| and
  \lstinline[basicstyle=\ttfamily]|setsockopt| to coax Linux into making both non-blocking
  sockets and non-blocking accepting of connections. This also has the interesting side effect of
  changing certain error codes' meaning, specifically \lstinline[basicstyle=\ttfamily]|EAGAIN| and
  \lstinline[basicstyle=\ttfamily]|EWOULDBLOCK|. Those two codes' meaning is now one of success, or not
  failing, rather than failure. Our error-handling code reflects that, which makes for some initial
  confusion if you are not familiar with the rationale behind it.
  \par\vspace{\baselineskip}
  As for the Work Manager's task of managing the worker threads that do the actual work, via the
  SQL Engine, of fulfilling the client's commands or queries, we used a thread pool concept that
  heavily relies on C++11 features. This thread pool is composed of three parts:
  \begin{itemize}
   \item A vector of \lstinline[]|std::future<Result>| objects
   \item A map that relates a job number to a connection, defined as:\\
	 \lstinline[]|std::map<uint32_t, omdb::Connection>|
   \item A vector of \lstinline[]|WorkThreadData| objects
  \end{itemize}
  The first part, the vector of \lstinline|std::future<Result>| objects, is
  essential to retrieving query results from the worker threads in an asynchronous way. This way
  we can continue to be non-blocking without needing to wait for a thread to finish which also
  allows us to be wait-free, our overriding requirement. Forcing the \lstinline|std::future<Result>|
  to be non-blocking when checking for a return value is currently clunky:
  \begin{lstlisting}[language=C++]
[&results] (std::future<Result>& res) -> bool
{
  auto time_out = std::chrono::seconds(0);
  // Typically, wait() blocks the thread,
  // but if we wait for a time of 0 it returns immediately
  if(res.valid() &&
     res.wait_for(time_out) == std::future_status::ready)
  {
    results.push_back(res.get());
    return true;
  }

  return false;
};
  \end{lstlisting}
  However, it does function in a way that is suitable for our reasons. Thankfully the rest of the thread
  pool is much smoother.
  \par\vspace{\baselineskip}
  The second part, the mapping of a job number to a connection, is needed to get the correct query results
  to its originating connection and thus the client that requested the data in the first place. While simple,
  implementing some sort of mapping or relationship similar to this makes a thread pool overly complicated
  or could lead to putting reference to the connection within the query result object itself, introducing
  complexity in that way.
  \par\vspace{\baselineskip}
  The vector of \lstinline[]|WorkThreadData| objects is the mechanism in
  which the Work Manager sends jobs, the queries or commands that a client sends to OpenMemDB, to the
  worker threads in the thread pool. The jobs are contained within a \lstinline[language=C++]|std::packaged_task<Result(int)>|
  object which is a C++11 construct that can generate the previously mentioned \lstinline|std::future<Result>|.
  The generated future object is stored in the afore-mentioned vector and the job is then passed to a selected
  worker thread to be evaluated. The selection process for where this job is placed is the work load distribution
  task that the Work Manager is responsible for. In order to do this accurately, the Work Manager implements
  uses a heuristic algorithm that estimates the computation needed to fulfill that query or command
  and then places the job in the thread pool in a way that balances overall system load.
  \par\vspace{\baselineskip}
  The heuristic used to guess the complexity of a given query or command is fairly simple as we need to
  be performant rather than accurate. The primary cost in this heuristic is length of the query/command itself
  as that is a relatively cheap operation, though some other costs could be the amount of parentheses
  or the actual classification of the query/command itself (i.e. whether it is a SELECT, INSERT INTO, or DELETE statement).
  This heuristic will be a source of heavy experimentation and testing as overloading a single core or thread
  with work can negatively affect the ability of the system to be wait-free as the rest of system is
  starved for work.
  \par\vspace{\baselineskip}
  \begin{figure}
   \centering
   % WorkManager class diagram
   \textbf{WorkManager Module Class Diagram}
   \includegraphics[scale=.80]{graphics/WorkManager_class.png}
   \caption{The class diagram of the main WorkManager class within OpenMemDB's Work Manager module}
  \end{figure}

  The flow of data from the client to the database and from there to the worker threads
  is as follows:
  \begin{figure}
   \centering
   % Data flow diagram (client -> wrk_mgr.assign_job)
   \textbf{Data Flow from Client to SQL Engine}
   \includegraphics[scale=.8]{graphics/WorkManager_activity.png}
   \caption{Shows the flow of data from a client to the worker threads within OpenMemDB}
  \end{figure}

  \subsection{Data Store}

  \subsection{Logger}
  %author Neil Moore
  Comprehensive logging is essential for any database that wants to be ACID, or
  at the very least persistent in the event of unexpected shutdown. However, we
  have a unique problem when implementing a logger in that we cannot access the
  hard disk. This is due to the fact that writing to hard disk is slow enough to
  impact our wait-freedom within the database, which is something we cannot afford.
  In order to be both non-blocking and persistent, at least in some rudimentary way,
  we are going to use a distributed model where queries that manipulate the data
  stored within a table (such as INSERT INTO, UPDATE, or DELETE) will be immediately
  sent back out over the network to a helper process.
  \par\vspace{\baselineskip}
  Although this doesn't entirely remove the problem, it does move it to an auxiliary
  program that doesn't have the same stringent requirements. This would allow us to
  perform disk writes or reads as needed. The recovery from a crash in the main database
  could be performed in much the same way that the logs were created. We would simply
  ``playback'' the queries contained within the log and that should lead us to the same
  state within the tables as it was when the database crashed.
  % this is the interaction diagram between the main database and the logger
  % should also include the interaction between database and client, just to be complete
  \begin{figure}
    \centering
    \textbf{Client, Database, and Logger Interaction Diagram}
    \includegraphics{graphics/Logger_interaction.png}
    \caption{The network interaction model employed by OpenMemDB to implement a
	     write-ahead-logger(WAL)}
  \end{figure}

\newpage

\section{Facilities and Equipment}
Our primary meeting place as a team was the computer science senior design lab in Harris Engineering
Center (HEC) 105. Workstations are provided along with collaboration equipment such as
whiteboards and computers hooked up to TVs. A rack-mounted server was also made available and
some members of the team were given virtual machines on it to use as development
platforms.
\par\vspace{\baselineskip}
Individual group members were able to use their personal machines as development platforms, though
eventually all but the most trivial testing had to be moved to the server(s) we had access to
as their consumer-level hardware could not provide enough raw power to properly benchmark and
stress-test the system. The hardware used by each member:
\begin{itemize}
 \item Thor (Neil Moore's personal machine)
 \begin{itemize}
  \item{\makebox[4cm]{CPU:\hfill} AMD FX-8350 (8-core)}
  \item{\makebox[4cm]{RAM:\hfill} 16 GB DDR3}
  \item{\makebox[4cm]{Motherboard:\hfill} Asus M5A99X EVO R2.0}
  \item{\makebox[4cm]{GPU:\hfill} AMD Radeon HD 7870}
  \item{\makebox[4cm]{Primary Hard Disk:\hfill} PNY Optima 240 GB SSD}
  \item{\makebox[4cm]{Total Storage:\hfill} 3 terabytes}
  \item{\makebox[4cm]{Operating System:\hfill} Arch Linux}
 \end{itemize}

 \item Mike McGee's personal machine
 \begin{itemize}
  \item{\makebox[4cm]{CPU:\hfill} Intel i7-4790K}
  \item{\makebox[4cm]{RAM:\hfill} 32 GB DDR3}
  \item{\makebox[4cm]{Motherboard:\hfill} GIGABYTE G1 Gaming G1.Sniper}
  \item{\makebox[4cm]{GPU:\hfill} GeForce GTX 970}
  \item{\makebox[4cm]{Primary Hard Disk:\hfill} Samsung 840 EVO 120GB SSD}
  \item{\makebox[4cm]{Total Storage:\hfill} 2 terabytes}
  \item{\makebox[4cm]{Operating System:\hfill} Windows 10 / Arch Linux VM}
 \end{itemize}

 \item Jason Stavrinaky's personal machine
 \begin{itemize}
  \item{\makebox[4cm]{CPU:\hfill} Intel i7-4770K}
  \item{\makebox[4cm]{RAM:\hfill} 8 GB DDR3}
  \item{\makebox[4cm]{Motherboard:\hfill} ASUS Z87-A}
  \item{\makebox[4cm]{GPU:\hfill} GeForce GTX 660 OC}
  \item{\makebox[4cm]{Primary Hard Disk:\hfill} Seagate 1 terabyte}
  \item{\makebox[4cm]{Operating System:\hfill} Windows 10 / Arch Linux dual boot}
 \end{itemize}

\end{itemize}
\par\vspace{\baselineskip}
In addition to the senior design lab, we also had access to the Scalable and Secure Systems lab server
thanks to Dr. Dechev. This server was the primary test-bed for OpenMemDB as consumer hardware simply
doesn't have the necessary parallization or memory needed in order to truly test and stress a
highly-parallelized and memory-intensive database management system. The server we were allowed to
use had the following hardware characteristics:
\begin{itemize}
 \item 64 cores
 \item 312 gigabytes of RAM
 \item 1 terabyte of hard disk storage
\end{itemize}

\newpage

\subsection{Consultants, Subcontractors, and Suppliers}
Throughout the course of designing and developing this database management system we sought the advice
and knowledge of various people, as none of us are experts or even particularly knowledgeable in
massively parallel systems. Our primary source of information and advice was Steven Feldman, the
original developer of Tervel and its maintainer, even past his departure from the University of Central
Florida to Google. His insight into the internals of Tervel proved invaluable, as did his many
explanations on the concepts of wait-freedom and how to create a wait-free system.
\par\vspace{\baselineskip}
Dr. Dechev also proved a reliable source to consult when either Steven was unavailable or when he
had particular insight into a problem we were facing.
%Include stuff about the other guy...
\newpage

\section{Budget and Financing}
Thanks to the contributions from our sponsor and the nature of our project, we did not need to expend
any additional funds outside of normal maintenance each team member performed on their personal
machines.
\newpage

\section{Project Summary}
\newpage

\bibliography{final_document}
\bibliographystyle{acm}
\newpage

\appendix
\section{Appendix A}


\end{document}
