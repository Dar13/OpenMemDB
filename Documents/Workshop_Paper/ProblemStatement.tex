


\documentclass[10pt]{article}

\usepackage[utf8]{inputenc}
\usepackage[pass, letterpaper]{geometry}
\usepackage{graphicx}


\begin{document}
	\pagenumbering{gobble}
	\section{Related Works}
	Databases are slow while memory is getting cheaper. Memory is following a long running pattern, from costing thousands of dollars for a few megabytes to about \$40 for 8 gigabytes \cite{jcmit}.
	With this trend, we can take advantage and design extremely fast databases.	While there are many applications for which	current database solutions are fast enough
	(website logins for example), these solutions leave something to be	desired	for countless other applications. Take big data analytics for example. With hundreds of terabytes of
	data and millions of users to	account for, traditional database	systems fail to provide real time analytics. 
	
	\subsection{The Zynga Case}Zynga, the developers of Words with Friends and Farmville, faced the slow database problem. With so many users, their database solution was not fast enough to provide real time analytics.
	There were just too many transactions going on at once. They remedied this by switching to MemSQL: a closed-source in-memory database. With this solution, Zynga can "make
	decisions based on billions of data points in real-time to provide better in-game personalization and overall customer satisfaction" \cite{MemSQL} and "make real-time
	in-game decisions based on players' engagement levels with new game extras or interactions with other players" \cite{MemSQL}.
	
	

	\section{Motivation}
	Our main motivation for this project is the desire to learn. Computer science is heading towards a very multiprocessor programming heavy era. Designing a wait-free database gives us the 		opportunity to learn more about multiprocessor software design. Our secondary objective for this project is to contribute to the computer science community. There is a lack of open-source, in-memory, and fully wait-free database systems. OpenMemDB plans to solve this problem. With our thirst for knowledge and motivation to contribute, we strive to improve the computer science community.

	\pagebreak

	\bibliography{ProblemStatement}
	\bibliographystyle{plain}

\end{document}
