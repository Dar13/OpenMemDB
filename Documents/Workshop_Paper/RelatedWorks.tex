\documentclass[10pt]{article}

\usepackage[utf8]{inputenc}
\usepackage[pass, letterpaper]{geometry}
\usepackage{graphicx}


\begin{document}
	\pagenumbering{gobble}
	\section{Related Works}
	Over the last 30 years the there has been tremendous advancements in computing
	hardware. "Processors are thousands of times faster and memory is thousands of
	times times larger"\cite{stonebraker2007end}. Most technologies have advanced 
	along with hardware, however database management systems have struggled to improve
	at a similar rate. This is mostly due to concurrency issues. "Existing studies show
	that current database engines can spend more than 30\% of time in 
	synchronization-related operations (e.g.locking and latching), even when only a 
	single client thread is running."\cite{soares2015database}
    \\\\
  	There have been several attempts to solve this problem. Some of which will sacrifice
  	some data consistency in order to achieve a higher better performance. Still others
  	remain fully ACID compliant and attempt to parallelize individual steps in the 
  	DBMS or solely use multiple threads when executing query plans. Then there are those
  	that attempt to implement some level of lock freedom into their DBMS.
  	\\\\
	\subsection{MemSQL and VoltDB}
	MemSQL\footnote{MemSQL can be configured as a Columnstore that stores data on disk}
	and VoltDB are both fully in memory DBMS as is OpenMemDB. This is where the
	similarities end as far as OpenMemDB is concerned. MemSQL and VoltDB on the other 
	hand both use distributed systems to achieve performance gains. MemSQL differs from 
	VoltDB in a few ways, the most important being it's use of lock free data structures
	for storing data and its storing of pre-compiled commonly used queries\cite{MemSQL}.
	VoltDB tries to make its performance gains by what they call "Concurreny through
	scheduling"\cite{VoltDB}. This is the process of using a single-threaded pipeline 
	that performs the task it was scheduled. This limits the need for locking during
	transactions by intelligently scheduling the transactions so that locks are not
	necessary.
	\\\\
	OpenMemDB aims to take a different approach, one that is fully wait free. The goal is 
	to use powerful wait free data structures that will allow for a massively parallel 
	DBMS that can scale with the addition of processors and memory. It is our assumption 
	that the achievement of a fully weight free DBMS will achieve the performance gains 
	that have been lacking in the DBMS world while eliminating the complexity of 
	distributed systems, all while retaining full ACID compliance.
	
	\pagebreak
	
	\bibliography{RelatedWorks}
	\bibliographystyle{plain}
	
\end{document}