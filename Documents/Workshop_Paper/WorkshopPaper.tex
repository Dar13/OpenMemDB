\documentclass[letter,11pt]{article}
%\documentclass[letter,10pt]{scrartcl}

\usepackage[utf8]{inputenc}
\usepackage[pass,letterpaper]{geometry}
\usepackage{graphicx}
\usepackage{listings}
\usepackage{color}
\usepackage{float}
\usepackage{bigfoot}
\usepackage[square,sort,comma,numbers]{natbib}
\usepackage{titling}
\usepackage{mathtools}
\usepackage{helvet}

\setlength{\parindent}{0cm}

\title{OpenMemDB: A wait-free database\thanks{Sponsor: Dr. Damian Dechev}}
\author{Michael McGee \and Robert Medina \and Neil Moore \and Jason Stavrinaky}
\date{2/1/2016}

\pdfinfo{%
  /Title    (OpenMemDB: A wait-free database)
  /Author   (Mike McGee, Neil Moore, Robert Medina, Jason Stavrinaky)
  /Creator  ()
  /Producer ()
  /Subject  ()
  /Keywords ()
}

\begin{document}
\pagenumbering{gobble}
\maketitle
\newpage

\pagenumbering{roman}
\tableofcontents
\newpage

\pagenumbering{arabic}

\section{Introduction}
While hard drives are getting faster, they are still relatively slow when compared 
with main memory while memory is getting cheaper. Memory is following a long running 
pattern where it has precipitiously dropped from costing thousands of dollars for a few 
megabytes to about \$40 for 8 gigabytes \cite{jcmit}. With this trend, we can take 
advantage and design extremely fast databases. While there are many applications for 
which current database solutions are fast enough (website logins for example), these 
solutions leave something to be desired for massively concurrent systems with high throughput
requirements such as real-time analytics on large datasets. While the largest of 
these datasets are still too large for main memory, some datasets are finally able
to fit into main memory in modern systems due to the cheapness and advancement of 
memory module technology. Making it possible for a data analysis algorithm to make 
full use of the database server's processing power to extract and manipulate data
would make that analysis quicker and efficient in terms of hardware use.
\par\vspace{\baselineskip}
A real-world example of this need for fast and high throughput databases is demonstrated by Zynga, 
the developers of Words with Friends and Farmville, faced a problem where their 
database was not able to supply their data analysis algorithm problem. With so 
many users, their database solution was not fast enough to provide real time analytics. 
There were just too many transactions going on at once. They 
remedied this by switching to MemSQL: a closed-source in-memory database. With this 
solution, Zynga can ``make decisions based on billions of data points in real-time to 
provide better in-game personalization and overall customer satisfaction''. \cite{MemSQL} 

\subsection{Motivation}

\section{Related Works}
Over the last 30 years the there has been tremendous advancements in computing
hardware. "Processors are thousands of times faster and memory is thousands of
times times larger"\cite{stonebraker2007end}. Most technologies have advanced 
along with hardware, however database management systems have struggled to improve
at a similar rate. This is mostly due to concurrency issues. "Existing studies show
that current database engines can spend more than 30\% of time in 
synchronization-related operations (e.g.locking and latching), even when only a 
single client thread is running."\cite{soares2015database}
\par\vspace{\baselineskip}
There have been several attempts to solve this problem. Some of which will sacrifice
some data consistency in order to achieve a higher better performance. Still others
remain fully ACID compliant and attempt to parallelize individual steps in the 
DBMS or solely use multiple threads when executing query plans. Then there are those
that attempt to implement some level of lock freedom into their DBMS.
\par\vspace{\baselineskip}
\subsection{MemSQL and VoltDB}
MemSQL\footnote{MemSQL can be configured as a Columnstore that stores data on disk}
and VoltDB are both fully in memory DBMS as is OpenMemDB. This is where the
similarities end as far as OpenMemDB is concerned. MemSQL and VoltDB on the other 
hand both use distributed systems to achieve performance gains. MemSQL differs from 
VoltDB in a few ways, the most important being it's use of lock free data structures
for storing data and its storing of pre-compiled commonly used queries\cite{MemSQL}.
VoltDB tries to make its performance gains by what they call "Concurreny through
scheduling"\cite{VoltDB}. This is the process of using a single-threaded pipeline 
that performs the task it was scheduled. This limits the need for locking during
transactions by intelligently scheduling the transactions so that locks are not
necessary.
\par\vspace{\baselineskip}
OpenMemDB aims to take a different approach, one that is fully wait free. The goal is 
to use powerful wait free data structures that will allow for a massively parallel 
DBMS that can scale with the addition of processors and memory. It is our assumption 
that the achievement of a fully weight free DBMS will achieve the performance gains 
that have been lacking in the DBMS world while eliminating the complexity of 
distributed systems, all while retaining full ACID compliance.

\newpage
\bibliography{WorkshopPaper}
\bibliographystyle{acm}
\newpage

\end{document}
